% !TEX root = 7390-notes.tex






% notes taken on 09-03-2013

\section{Jacobians and abelian varieties}

%\subsection*{Abelian varieties over $\dC$}

At first, we will work over $\dC$ and treat curves as Riemann surfaces. By 
\cite[I.6.12]{ha77} and \cite[5.8.7]{jo06}, 
the category of smooth projective curves over $\dC$ is equivalent to the 
category of compact connected Riemann surfaces, there is no loss of 
information here. Let's start with some general definitions.

\begin{definition}
A \emph{variety} over a field $k$ is a separated scheme of finite type over 
$\spec(k)$. We call a variety $X$ over $k$ \emph{nice} if it is smooth, projective, 
and geometrically integral.
\end{definition}

Recall that $X/k$ is \emph{geometrically integral} if 
$X_{\bar k}=X\times_k \spec(\bar k)$ is integral. A \emph{curve} is a 
variety of dimension one. If we are interested in $C(k)$ for general curves, 
it is sufficient to consider nice curves. If $X$ is a variety, we can consider 
its functor of points $h_X:\mathsf{Alg}_k\to \mathsf{Set}$ which assigns to a 
$k$-algebra $A$ the set $X(A)$ of ``$A$-valued points.'' This extends to a 
functor $h_X:\mathsf{Sch}_k^\circ\to\mathsf{Set}$ which is defined by 
$h_X(Y)=\hom_k(Y,X)$. 

Earlier, for $f\in \dQ[x,y]$ and $C$ defined by the equation $f=0$, we 
defined $C(A)=\{(a,b)\in A^2:f(a,b)=0\}$ for any $\dQ$-algebra $A$. 
Note that 
\begin{align*}
  C(A) &= \{(a,b)\in A^2:f(a,b) = 0\} \\
    &= \hom_{\mathsf{Alg}_\dQ}(\dQ[x,y]/f,A) \\
    &= \hom_{\mathsf{Sch}_k}\left(\spec A,\spec(\dQ[x,y]/f)\right) \text{,}
\end{align*}
so this agrees with our general definition. By the Yoneda Lemma, the functor 
$h_X$ determines $X$ up to isomorphism. 






\subsection{Jacobians over \texorpdfstring{$\dC$}{C}}

For the rest of this section, let $C$ be a nice curve over $\dC$. Set 
$X=C(\dC)$; this is a compact connected Riemann surface. So 
topologically, $X$ is a many-handled torus. Let $\Lambda=\h_1(X,\dZ)$, 
the first singular homology group of $X$, which can be identified with 
$\pi_1(X)^\text{ab}$. We will regard elements of $\Lambda$ are equivalence 
classes $[\gamma]$ for $\gamma\in \pi_1(X)$. 


It is a theorem of algebraic topology that $\Lambda\simeq\dZ^{2 g}$ for some 
$g\geqslant 0$; we 
will call $g$ the \emph{genus} of $X$ (and also of $C$). Let $K$ be the field 
of meromorphic functions on $X$. For $f\in K$, at any point 
$x\in X$, we can write $f$ in local coordinates as $z^n(a_0+a_1 z+\cdots)$ 
where $a_0\ne 0$ and $n\in\dZ$. We call $n=\ord_x(f)$ the \emph{order 
of vanishing} of $f$ at $x$. The \emph{degree} of $f$ is the sum 
$\deg(f)=\sum_{x\in X}\ord_x(f)$. If $f$ is holomorphic instead of just 
meromorphic, then $\deg(f)=0$ because $f$ is constant, but the converse fails. 
 
We can identify $K$ with the function field of $C$, i.e.\ the set of rational 
maps $C\to\dA^1$. Certainly rational maps yield meromorphic functions, 
and it is a basic theorem of Riemann surface theory that meromorphic functions 
are in fact algebraic. Moreover, if $C$ is nice, then $C$ can be recovered 
from $K$. To do this, pick some $x\in K\smallsetminus \dC$. If we let $A$ be 
the integral closure of $\dC[x]$ in $K$, then $\spec(A)$ will be a 
smooth affine curve. Pick some embedding of $\spec(A)$ into projective space; 
the closure of its image will be a projective curve $C'$ (possibly with 
singularities) with function field $K$. We can resolve the singularities of 
$C'$ to obtain a smooth projective curve $C''$ with function field $K$. By 
\cite[I.6.12]{ha77}, this induces an anti-equivalence between the category of 
extensions $K/\dC$ of transcendence degree one and the category of nice 
curves over $\dC$ with surjective morphisms. 

One might ask whether the singular homology $\h_1(X,\dZ)$ can be 
defined ``algebraically.'' Essentially, the answer is no -- that is, there is 
no known algebraic definition for $\h_1(X,\dZ)$ that gives the 
``right'' answers. On the other hand, $\h_1(X,\dQ)$ is naturally 
isomorphic to the dual of the algebraic de Rham cohomology 
$\h_\text{dR}^1(X/\dQ)$, and $\h_1(X,\dZ_\ell)$ is naturally 
isomorphic to the dual of the $\ell$-adic cohomology 
$\h_\text{et}^1(X,\dZ_\ell)$. Both of these isomorphisms are hard 
theorems -- the first due to Grothendieck \cite{gr66}, the 
second originally due to Artin, and proved in \cite[I 4.6.3]{de77}. 

With $C$ as before, let $V=\Omega^1(X)=\h^0(X,\Omega^1)=\h^1_\text{dR}(X)$ be 
the first analytic de Rham cohomology of $C$. 
This is a complex vector space of dimension $g$, so we get a non-topological 
definition of $g$. We can consider $\Omega^1$ as the sheaf of (algebraic) 
differentials, and $g=\dim_\dC\h^0(X,\Omega^1)$, giving us a purely algebraic 
definition of $g$. There is a natural pairing 
$\h_1(X,\dZ)\otimes\h_\text{dR}^1(X)\to\dC$, defined by 
\[
  [\gamma]\otimes \omega \mapsto \int_\gamma \omega
\]
This pairing is nondegenerate and $\dC$-linear in the second component, 
It induces an injection $\Phi:\Lambda\hookrightarrow V^\vee$. 

\begin{definition}[analytic]
The \emph{Jacobian} of $X$ is $\jac(X)=V^\vee/\Phi(\Lambda)$. 
\end{definition}

Is is known that $\Phi(\Lambda)$ is a lattice in $V^\vee$, i.e. it is discrete 
and $V^\vee/\Phi(\Lambda)$ is compact. There is the \emph{Abel-Jacobi map} 
$j:X\to \jac(X)$ defined as follows. Fix $x_0\in X$; we send $x\in X$ to 
$\omega\mapsto \int_{[x_0,x]} \omega$, where $[x_0,x]$ denotes some path from 
$x_0$ to $x$. A different choice of $[x_0,x]$ will differ by a closed loop, 
i.e. an element of $\h_1(X,\dZ)$. So $j:X\to\jac X$ is well-defined. Note 
that $\jac X$ is a compact complex Lie group. 

After choosing a basis for $V^\vee$, we have $\jac X\simeq \dC^g/L$, where 
$L\simeq \dZ^{2 g}$. As a real Lie group, $\jac X$ is isomorphic to 
$(S^1)^{2 g}$. 
We care about $\jac X$ because, despite its analytic definition, it is in fact 
a projective variety.

\begin{theorem}
For $X$ a compact Riemann surface, $\jac X$ is algebraic, i.e. there exists a 
variety $J$ defined over $\dC$ such that $J(\dC)\simeq\jac X$ as 
complex manifolds. Moreover, the group operation on $\jac X$ is algebraic, 
i.e. there is a morphism $m:J\times_\dC J\to J$ such that 
$J(\dC)\times J(\dC)\to J(\dC)$ corresponds to the 
addition law on $\jac X$. 
\end{theorem}
\begin{proof}
See \cite[I.18]{mi-av}. Essentially, $X$ is the analytification of a curve $C$, 
and one proves that $\jac X$ (defined analytically) is isomorphic as a complex 
manifold to the analytification of $\jac C$ (defined algebraically). 
\end{proof}

Let $\divisor X$ be the free abelian group generated by the points of $X$. There 
is a map $\deg:\divisor X\to \dZ$, defined by 
$\sum n_x\cdot x\mapsto \sum n_x$. We define $\divisor^\circ X$ by the short exact 
sequence 
\[\xymatrix{
  0 \ar[r]
    & \divisor^\circ X \ar[r]
    & \divisor X \ar[r]
    & \dZ \ar[r]
    & 0 \text{.}
}\]
There is also a map $\operatorname{div}:K^\times\to \divisor X$, where 
$\operatorname{div}(f) = \sum_x \ord_x(f)\cdot x$. It is a basic fact that 
$\deg(\operatorname{div}(f)) = 0$, so we can define the \emph{Picard group} 
of $X$ to be $\picard X = \divisor X/\operatorname{div}(K^\times)$ and 
$\picard^\circ X = \divisor^\circ (X)/\operatorname{div}(K^\times)$. 

Let $\mathscr{M}$ be the sheaf of meromorphic functions on $X$. It is a basic 
fact that $\divisor(X)=\h^0(X,\mathscr{M}^\times/\sO^\times)$ and 
$\picard(X)=\h^1(X,\sO^\times)$. Indeed, the first equality is often taken to 
be a definition as in \cite[III.6]{ha77}, and the second is a straightforward 
exercise in \v Cech cohomology. 
An example where the Picard group is easily 
determined is $\picard^\circ(\dP^1) = 0$.

The Abel-Jacobi map $j:X\to \jac X$ extends naturally to a homomorphism  
$j:\divisor^\circ X\to \jac X$. 

\begin{theorem}[Jacobi]
The map $j:\divisor^\circ X\to\jac X$ is surjective.
\end{theorem}

\begin{theorem}[Abel]
The kernel of $j:\divisor^\circ X\to \jac X$ is $\operatorname{div}(K^\times)$. 
\end{theorem}

It follows that $j$ induces an isomorphism $j:\picard^\circ X\xrightarrow\sim \jac X$. 
Note that $\picard^\circ X$ parameterizes invertible sheaves (line bundles) on 
$X$ of degree zero. 

Note that in general, $\dC^g/L$ for some lattice $L$ need not be 
algebraic if $g>1$. In the future, we'll try to define $\jac C$ for a curve 
$C$ over any field. The variety $\jac C$ will be a nice variety, i.e. 
smooth, projective and geometrically integral. We will use this to give an 
``algebraic definition of $\h_1(X,\dZ/n)$.'' 





% lecture on 09-05-2013
\subsection{Abelian varieties over arbitrary fields}

Recall that a variety $X/k$ is \emph{nice} if it is smooth, projective, and 
geometrically connected. 

\begin{definition}
Let $k$ be a field. An \emph{abelian variety} over $k$ is a nice group variety 
over $k$.
\end{definition}

In other words, there are morphisms $m:A\times A\to A$, $i:A\to A$, 
$e:\spec(k)\to A$ such that the induced maps $m_*:h_A\times h_A\to h_A$ etc. 
turn $h_A$ into a group-valued functor. In particular, $A(X)$ is an ``honest  
group'' for each $k$-scheme $X$. 

\begin{example}
The general linear group $\operatorname{GL}(n)$ is a group variety, but not 
nice (at least, not in the technical sense) because it is not projective. More 
generally, no linear algebraic group is an abelian variety, for the same 
reason. 
\end{example}

Note that abelian varieties are not required to be commutative, but this is in 
fact the case. This is easy to see over $\dC$. If $A/\dC$ is an 
abelian variety, then $G=A(\dC)$ is a compact connected complex Lie 
group. Let $\mathfrak{g}$ be its Lie algebra. Consider the composite map 
$f:G\xrightarrow{\text{ad}}\operatorname{GL}(\mathfrak{g}) \hookrightarrow 
\End(\mathfrak{g})$, where 
$\operatorname{ad}:G\to\operatorname{GL}(\mathfrak{g})$ is the adjoint 
representation. After picking a basis for $\End(\mathfrak{g})$, 
the components of $f$ are entire holomorphic functions on a compact complex 
manifold, hence locally constant. Since $G$ is connected, $f$ is constant, i.e. 
the adjoint representation of $G$ is trivial. But 
$\ker(\operatorname{ad}) = Z(G)$, so $G$ is commutative.
The case $A/k$ for arbitrary $k$ of characteristic zero 
follows from the Lefschetz principle, or one can just prove commutativity 
directly using a ``rigidity principle'' for maps on projective varieties
\cite[I.1.4]{mi-av}. 





\subsection{Albanese varieties}

We'd like to describe the jacobian $J$ of a nice curve $C/k$ with 
$C(k)\ne\varnothing$. It will be an abelian variety over $k$ of dimension 
$g$, the genus of $C$. So far we've only defined the genus of a curve $C/k$ with 
$k\subset \dC$. For an arbitrary field $k$ and a curve $C/k$, set 
$g(C)=h^0(C_{\bar k},\Omega^1) = h^1(\sO_C)$. In general, if $\sF$ is some 
sheaf on a scheme $X$ over $k$, we write $h^i(X,\sF)$ or $h^i(\sF)$ for 
$\dim_k \h^i(X,\sF)$.

\begin{definition}[Albanese]
Let $C/k$ be a curve with fixed $x_0\in C(k)$. The \emph{jacobian} of $C$ is 
an abelian variety $J=\jac C$ with a morphism $j:C\to J$ taking $x_0$ to $0$, such 
that for any morphism $f:C\to A$ to an abelian variety $A$ with $f(x_0)=0$, 
there is a unique $\tilde f:J\to A$ making the following diagram commute:
\[\xymatrix{
  C \ar[r]^-j \ar[dr]_-f 
    & J \ar@{.>}[d]^-{\tilde f} \\
  & A
}\]
\end{definition}

Since $(J,j)$ is the solution of a universal problem, it is unique up to 
unique isomorphism. Our definition can be made much more concise. Let 
$\mathsf{AbV}_k$ be the category of abelian varieties over $k$, and let 
$\mathsf{Var}_{k,\ast}$ be the category of ``nice pointed varieties'' over $k$, 
i.e. nice varieties $X/k$ with chosen $x\in X(k)$. Forgetting the group 
structure gives an inclusion functor 
$\iota:\mathsf{AbV}_k\to \mathsf{Var}_k$. Our definition of $\jac C$ can be 
rephrased as saying that $j:C\to \jac C$ induces a natural isomorphism 
\[
  \hom_{\mathsf{Var}_{k,\ast}}(C,\iota A) \simeq \hom_{\mathsf{AbVar}_k}(\jac C,A)\text{.}
\]
It turns out that for any nice pointed variety $X$, there is an abelian variety 
$A=\operatorname{Alb} X$, the \emph{Albanese variety} of $X$, with a morphism 
$j:X\to \operatorname{Alb} X$ that induces a similar natural isomorphism (with 
$\operatorname{Alb} X$ in place of $\jac C$). So taking jacobian may be seen 
as the left-adjoint to the forgetful functor from abelian varieties to pointed 
varieties. For a proof that $\operatorname{Alb} X$ exists, see 
\cite[A.11]{mo12}.

In general, the map $X\to \operatorname{Alb} X$ need not be an embedding. 
For example, if $C$ is a curve of genus $0$, then $\operatorname{Alb} X=0$ by 
\cite[I.3.9]{mi-av}. On the other hand, if the genus $g\geqslant 1$, then 
$C\to J$ is an embedding. The map $C(\bar k)\to \picard^\circ(C_{\bar k})$ sends 
a point $x$ to the divisor $[x]-[x_0]$. If $[x]-[x_0] = [y]-[x_0]$ in 
$\picard^\circ(C_{\bar k})$, then $[x]-[y] = \operatorname{div}(f)$ for some 
rational map $f:C\to \dP^1$. If $x\ne y$, then $f$ would have a unique simple 
zero and poll, which would imply that $f$ is birational. But this is 
impossible, so $x\ne y$ implies $j(x)\ne j(y)$. 

If $C(k)=\varnothing$, we can still define $J=\jac C$. It will be a $k$-variety 
with a morphism $j:C\times C\to J$ such that $j(\Delta)=0$. We require 
that for any abelian variety $A$ over $k$ with $f:C\times C\to A$ such that 
$f(\Delta)=0$, there is a unique lift $\tilde f:J\to A$ of $f$. The map $j$ 
should be thought of ``$(x,y)\mapsto j(x)-j(y)$,'' even though an embedding 
$C\to J$ may not be defined over $k$. 





\subsection{Picard schemes}\label{sec:picard-scheme}

Another approach to defining $J=\jac C$ involves the Picard group. Recall 
that over $\dC$, we proved that $J(\dC)\simeq\picard^\circ(C)$. 
One might hope that $J$ satisfies $J(L)=\picard^\circ(C_L)$ for all field 
extensions $L/k$. This works if $C(k)\ne\varnothing$, but not otherwise. For, 
if $J(k^s)=\picard^\circ(C_{k^s})$, then we would have 
$\picard^\circ(C) = J(k) = J(k^s)^{G_k} = \picard^\circ(C_{k^s})^{G_k}$. But this 
does not always hold. 

For a curve $C$ over $k$, we will define an abelian variety $\picard^\circ(C)$ in 
terms of its functor of points. To do this, we need to define Picard groups for 
arbitrary schemes. For any scheme $X$, the \emph{Picard group} of $X$ is 
$\picard(X)=\h^1(X,\sO_X^\times)$. It is straightforward to show (using \v Cech 
cohomology) that $\picard(X)$ is isomorphic to the group of isomorphism classes of 
invertible sheaves on $X$, with group operation induced by tensor product, i.e. 
$[\sL] + [\mathscr{L'}] = [\sL\otimes \sL']$. 

Let $C$ be a nice curve. If $D=\sum D_x\cdot x$ is a divisor on $C$, the 
\emph{degree} of $D$ is $\deg D=\sum D_x\in \dZ$. Since $C$ is smooth, Cartier 
divisors are Weil divisors, so $\deg$ induces a well-defined map 
$\picard(C)\to \dZ$. For $T$ an arbitrary scheme, define 
$\picard^\circ(C\times T)$ to be the subset of $\picard(C\times T)$ consisting of 
invertible sheaves $\sL$ with $\deg(\sL_t)=0$ for all $t\in T$. That is, the 
following sequence is exact 
\[\xymatrix{
  0 \ar[r] 
    & \picard^\circ(C\times T) \ar[r] 
    & \picard(C\times T) \ar[r] 
    & \displaystyle\prod_{t\in T} \dZ \text{,}
}\]
where the last map is $c\mapsto (\deg(c_t))_{t\in T}$. Now we define the 
functor $\picard_C^\circ:\mathsf{Sch}_k^\circ\to\mathsf{Ab}$ that sends $T$ to 
$\picard^\circ(C\times_k T)/\picard(T)$. 

\begin{theorem}
If $C(k)\ne \varnothing$, then $\picard_C^\circ$ is represented by $J=\jac C$.
\end{theorem}
\begin{proof}
See \cite[III.1.2]{mi-av}.
\end{proof}

In general, we might have $C(k)=\varnothing$. We will have 
$J(L)=\picard^\circ(C_{L'})^{\gal(L'/L)}$ where $L'/L$ is any separable extension 
with $C(L')\ne\varnothing$. 
Let $J=\jac C$ and $j:C\to J$ be the standard embedding. Let 
$C^g=C\times\cdots\times C$ ($g$-fold product), and consider the map 
$f:C^g\to J$, $f(x_1,\dotsc,x_g)=j(x_1)+\cdots+j(x_g)$. The symmetric 
group $S_g$ acts on $C^g$, and $f$ is $S_g$-equivariant. The quotient 
$\operatorname{Sym}^g C = C^g/S_g$ exists, and has a map 
$\operatorname{Sym}^g C\to J$. This is birational. Weil defined a ``rational 
group law'' on $\operatorname{Sym}^g C$ using the Riemann-Roch theorem, and 
then showed that this induces an ``honest group law'' on a nice variety 
birational to $\operatorname{Sym}^g C$. For 
more details on Weil's construction (and proofs), see \cite[III.7]{mi-av}. 



% lecture on 09-10-2013
Now suppose $X$ is an arbitrary scheme. Recall that 
$\picard(X)=\h^1(X,\sO_X^\times)$; this classifies invertible sheaves on 
$X$, where the group operation on sheaves is $\otimes$. If $X$ is integral, 
$\picard(X)$ is isomorphic to the class group $\operatorname{Cl}(X)$ of Cartier 
divisors (Weil divisors if $X$ is a nice curve). This is easy to prove. Recall 
that if $\mathscr{M}$ is the sheaf of rational functions on $X$, then the 
group of Cartier divisors is $\divisor(X) = \h^0(X,\mathscr{M}^\times/\sO^\times)$. 
The short exact sequence 
\[
  1 \to \sO^\times \to \mathscr{M}^\times\to \mathscr{M}^\times/\sO^\times \to 1
\]
induces a long exact sequence in sheaf cohomology:
\[
  0 \to \h^0(\sO^\times) \to \h^0(\mathscr{M}^\times) \to \divisor(X) \to \picard(X) \to \h^1(\mathscr{M}^\times) \to \cdots
\]
If $X$ is integral, the sheaf $\sM^\times$ is flasque, so $\h^1(\sM^\times)=0$. 
It follows that 
$\picard(X)=\divisor(X)/\h^0(\sM^\times) = \operatorname{Cl}(X)$. 

For $X/k$ an arbitrary scheme, consider the functor 
$\picard_X:\mathsf{Sch}_k^\circ\to\mathsf{Ab}$ given by 
$\picard_X(T) = \picard(X\times_k T)/\picard(T)$. This is not in general 
representable. However, if $X$ is a nice $k$-variety, then the 
fppf-sheafification of $\picard_X$ is representable \cite[4.1.38]{kl05}. Even 
better, if $X(k)\ne\varnothing$, then $\picard_X$ is representable 
\cite[2.5]{kl05}. We will also denote the representing scheme by $\picard_X$, and 
we call $\picard_X$ the \emph{Picard scheme} of $X$. It is not a variety, but it 
is a disjoint union of ind-varieties \cite[4.8]{kl05}. More precisely, choose 
a very ample line bundle $\sL$ on $X$. If $\sF$ is any coherent 
sheaf on $X$, write 
$\sF(n) = \sF\otimes\sL^{\otimes n}$. Recall that the 
\emph{Euler characteristic} of $\sF$ is 
$\chi(\sF) = \sum (-1)^i h^i(\sF)$. By \cite[2.5.3]{gr61}, 
there is a (unique) polynomial $\phi\in \dQ[t]$ such that 
$\chi\left(\sF(n)\right) = \phi(n)$ for all 
$n\in\dZ$; set $h_\sL(\sF) = \phi$. We call $\phi$ the 
\emph{Hilbert polynomial} of $\sF$. The Hilbert polynomial 
$h_\sL(\sF)$ does depend on $\sL$.

If $X(k)\ne \varnothing$, we can define for $x\in X(k)$ the modified Picard 
functor 
\[
  \picard_{X,x}(T) = \{(\sL,i):\sL\in \picard(X_T)\text{ , }i:x_T^\ast \sL \xrightarrow\sim \sO_T\} /\sim \text{.}
\]
There is an obvious map $\picard(X_T) \to \picard_{X/x}(T)$ given by 
$\sL\mapsto\sL\otimes (x\circ f)_T^\ast \sL^{-1}$, where $f:X\to \spec(k)$ 
denotes the structure morphism. It is a good exercise to prove that this 
induces an isomorphism $\picard_X \xrightarrow\sim \picard_{X,x}$. 

Denote by 
$\picard_X^\phi$ the functor which assigns to a scheme $T$ the subset of 
$\picard_X(T)$ consisting of invertible sheaves $\sF$ on $X\times T$ with 
$h_\sL(\sF_t) = \phi$ for all $t\in T$. By \cite[6.20]{kl05}, 
$\picard_X^\phi$ represents a clopen subscheme of $\picard_X$, and $\picard_X$ is 
covered by the $\picard_X^\phi$. Moreover, the $\picard_X^\phi$ are varieties. We can 
do even better. If we let $\picard_X^d$ send $T$ to the subset of $\picard_X(T)$ 
consisting of invertible sheaves $\sF$ with 
$\deg h_\sL(\sF_t) = d$ for all $t\in T$, then the 
$\picard_X^d$ form a cover of $\picard_X$ by clopen subvarieties. Just as the genus 
of a curve is the dimension of its jacobian, there is a natural isomorphism 
$\h^1(\mathscr{O}_X) \simeq \operatorname{Lie}(\picard_X)$, from which we deduce 
$\dim(\picard_X) = h^1(\mathscr{O}_X)$ when $X$ is nice \cite[5.11]{kl05}.

Unlike the case when $X$ is a curve, it is not always true that 
$\picard_X(\bar k)/\picard_X^\circ(\bar k) = \dZ$. In general, we set 
$\operatorname{NS}(X) = \picard_X(\bar k) / \picard_X^\circ(\bar k)$, and call 
$\operatorname{NS}(X)$ the \emph{N\'eron-Severi group} of $X$. 
Suppose $X=A$ is already an abelian variety over $k$. Then we have 
\[
  \picard_A^\circ(\bar k) = \left\{c\in \picard(A_{\bar k}) : t_a^* c = c\text{ for all }a\in A(\bar k)\right\} 
\]
where $t_a:A_{\bar k}\to A_{\bar k}$ is translation by $a$. See 
\cite[I.8.4]{mi-av} for a partial proof. 

For an abelian variety $A$ over $k$, 
the \emph{dual} of $A$ is defined to be $A^\vee = \picard_A^\circ$. 
Each $c\in \picard(A)$ gives a map $\varphi_c:A\to A^\vee$. At the level of 
points, it is defined as $a\mapsto t_a^* c - c$, where if $c=[\sL]$, 
the class $t_a^*c - c\in A^\vee(\bar k) = \picard^\circ(A)$ is represented by 
$[t_a^*\sL\otimes\sL^{-1}]$. 
It turns out that $A^{\vee\vee} \simeq A$, so calling $A^\vee$ the dual of $A$ 
is rather natural. The map $\varphi_c:A\to A^\vee$ is a homomorphism of abelian 
varieties. If $c$ is ample (i.e. the map from $A$ to some projective space 
induced by $n\cdot c$ for $n\gg 0$ is an embedding) then $\varphi_c$ is an 
\emph{isogeny}, where 

\begin{definition}
A homomorphism $\varphi:A\to B$ is an \emph{isogeny} if it is surjective with 
finite kernel.
\end{definition}

It is not at all obvious, but ``$A$ is isogenous to $B$'' is an equivalence 
relation on abelian varieties. The relation is clearly reflexive and 
transitive. To see that it is symmetric, suppose we have an 
isogeny $\varphi:A\to B$. For any ample $c\in \picard(A^\vee)$, $d\in \picard(B)$, 
the composite 
\[\xymatrix{
  B \ar[r]^-{\varphi_d} 
    & B^\vee \ar[r]^-{\varphi^\vee} 
    & A^\vee \ar[r]^-{\varphi_c}
    & A^{\vee\vee} \ar[r]^-\sim 
    & A
}\]
is an isogeny. 

\begin{definition}
Let $A$ be an abelian variety. A \emph{polarization} of $A$ is an isogeny 
of the form $\varphi_c:A\to A^\vee$ for some ample $c\in \picard(A)$.
\end{definition}

The duality theory of abelian varieties is very rich. A good place to start 
is Chapter VII of \cite{gm13}. 





\subsection{Recovering a curve from its jacobian}

Let $k$ be a field, $C/k$ a nice curve, and $J=\jac C$ its jacobian. What does 
(the arithmetic of) $J$ tell us about (the arithmetic of) $C$? In particular, 
can we recover $C$ from $J$? In general, $J$ does not determine $C$. For 
example, if $g=g(C)=0$, then $J=0$. However, there are (non-algebraically 
closed) fields $k$ for which there are nice curves $C$ over $k$ with $g(C)=0$ 
(hence $\jac C=0=\jac\dP^1$), but $C\not\simeq \dP_k^1$. There are more 
difficult examples with genus $g>0$. 

Suppose we add some data. Assume $g\geqslant 2$ and $C(k)\ne\varnothing$. This 
gives us a map $j:C\to J$ determined by $x\mapsto 0$ for some distinguished 
$x\in C(k)$. Consider $\theta = j(C)+\cdots + j(C)$, where there are $g-1$ 
terms in the sum. It turns out that $\theta$ is an irreducible ample divisor of 
$J$. Thus $\theta$ induces a polarization $\varphi_\theta:J\to J^\vee$. 

\begin{theorem}[Torelli]
If $C,C'$ are nice curves over a field $k$ with $(J,\theta)\simeq (J',\theta')$, 
then $C\simeq C'$ over $k$. 
\end{theorem}
\begin{proof}
See \cite[III.13]{mi-av} for a rather unenlightening proof. 
\end{proof}

It is known that $\varphi_\theta:J\to J^\vee$ is actually an isomorphism. We 
call a polarization $A\to A^\vee$ \emph{principal} if it is an isomorphism. 
The famous \emph{Schottky problem} asks 
what pairs $(A,c)$, with $c$ inducing a principal polarization, come from 
jacobians. 

This question has an easy partial answer if we are willing to use heavy 
machinery. Consider the functor $\mathcal{M}_g$ which sends a scheme $S$ to the 
set of isomorphism classes of curves of genus $g$ over $S$. The functor 
$\mathcal{M}_g$ is unfortunately not representable (there is no  
\emph{fine} moduli spaces for curves), but it is nearly so (there is a 
\emph{coarse} moduli space). That is, there is a scheme (a variety, 
actually) $M_g$ together with a natural transformation 
$\mathcal{M}_g\to h_{M_g}$ such that $\mathcal{M}_g(L)\to h_{M_g}(L)$ is a 
bijection whenever $L$ is an algebraically closed field, and such that 
$\mathcal{M}_g\to h_{M_g}$ is initial among all morphisms from $\mathcal{M}_g$ 
to representable functors. For a proof, see \cite[5]{mu94}. Along the same lines, 
we can let $\mathcal{A}_g$ be the functor which assigns to a scheme $S$ the set 
of isomorphism classes of principally polarized abelian schemes of dimension 
$g$ over $S$. The functor $\mathcal{A}_g$ has a coarse moduli space $A_g$. 

The operation ``take jacobian with its canonical polarization'' induces a natural transformation 
$j:\mathcal{M}_g\to\mathcal{A}_g$, and Torelli's theorem can be rephrased as 
saying that $j$ is injective. Shottky's question asks what the image of $j$ is. 
To see that it cannot be all of $\mathcal{A}_g$, simply note that 
$\dim(M_g) = 3 g - 3$, while $\dim(A_g) = \frac{g(g+1)}{2}$. For 
$g>3$, $A_g$ has greater dimension than $M_g$, so 
$j:M_g(\bar\dQ)\to A_g(\bar\dQ)$ cannot possibly be surjective. On the other 
hand, for $g\leqslant 3$, all principally polarized abelian varieties are 
jacobians (possibly after a change of the polarization). 

One might hope that all abelian varieties are at least isogenous to jacobians. 
While this is true for dimension $d\leqslant 3$, it is not true in general. 
In fact, for all $d>3$, there exists an abelian variety of dimension $d$ over 
$\overline{\dQ}$ which is not isogenous to a jacobian. This was proven 
recently in \cite{ts12}

\begin{example}[group law on elliptic curves]
Let $k$ be field of characteristic not $2$ or $3$. Let $E$ be an elliptic curve 
of the form $y^2=x^3+a x+b$ with $4 a^3+27 b^2\ne 0$. That is, $E$ is the 
subset of $\dP_k^2$ given by 
\[
  x_1^2 x_2 - x_0^3 - a x_0 x_2^2 - b x_2^3 \text{.}
\]
The choice of $O=(0:1:0)\in E(k)$ induces an embedding $j:E\to \jac(E)$ which 
is an isomorphism by the Riemann-Roch theorem. We would like to relate the 
induced group operation on $E$ with the classical definition using chords and 
tangents. 

Let $P,Q\in E(\bar k)$. If we assume $P,Q\ne O$, then we can write 
$P=(P_0:P_1:1)$, $Q=(Q_0:Q_1:1)$. Assume $P_0\ne Q_0$. Then there is an obvious 
rational function (canonical up to scale), whose zero-set is a 
line containing both $P$ and $Q$. Indeed, we put 
\[
 \ell_{P,Q}(x_0:x_1:x_2) = \frac{Q_1-P_1}{Q_0-P_0} \cdot \frac{x_0}{x_2} - \frac{x_1}{x_2} + \frac{P_1 Q_0 - P_0 Q_1}{Q_0-P_0} \text{.}
\]
The function $\ell_{P,Q}$ has $P$, $Q$ and a third point $R$ as simple zeros, 
and one can verify directly that 
$\operatorname{div}(\ell_{P,Q})=P+Q+R-3 O = (P-O)+(Q-O)+(R-O)$. Recall that 
$(\jac E)(\bar k)=\picard^\circ(E_{\bar k})$, and the map $j:E\to \jac(E)$ 
corresponds to $P\mapsto P-O$. Thus 
$j(P)+j(Q)+j(R)=\operatorname{div}(\ell_{P,Q})=0$ in 
$\picard^\circ(E_{\bar k})$, i.e. $j(P)+j(Q)=-j(R)$. 

It is well-known that $R$ can be written as a rational function of $P$ and $Q$, 
so the chord-tangent law defines a rational map $m:E\times E\to E$ with 
$j(m(P))+j(m(Q))=m(j(P),j(Q))$. It follows that $m$ is defined everywhere. 
\end{example}








