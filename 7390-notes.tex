\documentclass{article}

\usepackage{amsmath,amssymb,amsthm,fullpage}
\DeclareMathOperator{\spec}{Spec}
\newcommand{\fa}{\mathfrak{a}}
\newcommand{\fb}{\mathfrak{b}}

\newtheorem{theorem}{Theorem}
\theoremstyle{definition}
\newtheorem{example}{Example}

\title{Arithmetic of curves}
\author{David Zywina\thanks{notes written by Daniel Miller}}
\date{fall of 2013}

\begin{document}
\maketitle










\section{Introduction}





Let $\mathsf{CRing}$ be the category of commutative, unital rings, and 
$\mathsf{Aff}$ be its opposite. We call an object of $\mathsf{Aff}$ an 
\emph{affine scheme}, and for $A$ a commutative ring, write $\spec(A)$ for the 
corresponding object in $\mathsf{Aff}$. Consider the topology on 
$\mathsf{Aff}$ for which the coverings are those generated by the families 
$\{\spec(A_{f_i})\to \spec(A)\}_i$, where $\{f_i\}$ is a family in $A$ 
generating the unit ideal. If $F\subset G$ are presheaves on $\mathsf{Aff}$, 
we say that $F$ is an \emph{open subfunctor} of $G$ if when $A$ is a 
commutative ring, the map $\spec(A)\times_F G\to \spec(A)$ is (up to 
isomorphism) of the form $\spec(A_f)\to \spec(A)$ for some $f\in A$. We call 
a sheaf $X$ on $\mathsf{Aff}$ a \emph{scheme} if it has an open cover by 
(the functors of points of) affine subschemes. Let $\mathsf{Sch}$ denote 
the category of schemes. 

\begin{example}
Define $\mathcal{O}(A)=A$, considered as a set. Then $\mathcal{O}$ is a scheme 
represented by $\spec(\mathbb{Z}[T]))$. 
\end{example}

It follows from faithfully flat descent that the topology we have put on 
$\mathsf{Aff}$ (usually called the \emph{Zariski topology}) is subcanonical. 
Thus all representable presheaves are in fact sheaves, so all affine schemes 
are schemes. 

\begin{example}
Define $\mathbb{P}^n$ by letting $\mathbb{P}^n(A)$ be the set of locally 
free rank-one quotients of $A^{n+1}$. So if $k$ is a field, then 
$\mathbb{P}^n(k)$ is the usual $n$-dimensional projective space over $k$. 
\end{example}

The general notion of a scheme is not especially useful, but fortunately we 
will almost exclusively be concerned with \emph{projective varieties} -- that 
is -- closed subschemes of 
$\mathbb{P}_k^n=\mathbb{P}^n\times_{\spec(\mathbb{Z})}\spec(k)$ for some $n$, 
where $k$ is a field. Note that we regard $\mathbb{P}_k^n$ as a sheaf on 
$\mathsf{Aff}_k$, the opposite to the category of $k$-algebras. In fact, we 
will usually be interested in curves, which for us will be smooth projective 
varieties of dimension one, but as not all those terms are defined, the 
definition does not make sense. 





\end{document}
