% !TEX root = 7390-notes.tex





\section{Some theorems of Faltings}





\subsection{Background and Tate's conjecture}

The goal of this section is to describe the relationships between a web of 
conjectures that Faltings proved in his groundbreaking paper 
\cite{fa86}. 

Let $A$ be a $d$-dimensional abelian variety over a field $k$. As usual, we 
write $\bar k$ for the algebraic closure of $k$ and $G_k=\gal(\bar k/k)$ for 
the absolute Galois group of $k$. Fix a prime $\ell$ invertible in $k$. The 
groups $A[\ell^n] = \{x\in A(\bar k):\ell^n x=0\}$ are abstractly isomorphic 
to $(\dZ/\ell^n)^{\oplus 2d}$, and carry a continuous action of $G_k$. They 
fit into an inverse system 
\[\xymatrix{
  A[\ell] 
    & A[\ell^2] \ar[l]_-\ell 
    & A[\ell^3] \ar[l]_-\ell 
    & \cdots \ar[l]
}\]
Put 
\[
  T_\ell A = \varprojlim A[\ell^n] = \left\{(x_n)\in \prod A[\ell^n] : \ell x_{n+1} = x_n\right\} \text{.}
\]
This is the \emph{$\ell$-adic Tate module of $A$}. As a $\dZ_\ell$-module, 
$T_\ell A \simeq \dZ_\ell^{\oplus 2 d}$. What makes $T_\ell A$ interesting is 
that it carries a continuous action of $G_k$, induced by the action of $G_k$ 
on the $A[\ell^n]$. In other words, after choosing a basis of $T_\ell A$, we 
have a continuous representation 
\[
  \rho_{A,\ell} : G_k \to \operatorname{GL}(2 d,\dZ_\ell) \text{.}
\]
In fact, the representation $\rho_{A,\ell}$ factors through the smaller group 
$\operatorname{GSp}(2 n,\dZ_\ell)$. One sees this via the \emph{Weil pairing} 
$e:T_\ell A \times T_\ell A^\vee \to \dZ_\ell(1)$. 

(See \cite{ta65} for original Tate conjectures.) 

$\dQ_\ell$-vector space. We should think of $V_\ell A$ as an algebraic 
analogue of $\h_1(A,\dQ_\ell)$. The group $G_k$ acts compatibly on the 
$A[\ell^n]$, hence it acts on $T_\ell A$. This action is continuous and 
respects the group structure on $T_\ell A$. Thus we have a continuous 
representation 
$\rho_{A,\ell} : G_k \to \aut_{\dQ_\ell}(T_\ell(A))\simeq GL(2 g,\dZ_\ell)$. 
(This actually lands inside $\operatorname{GSp}(2 g,\dZ_\ell)$, via the Weil 
pairing. \textbf{add details}). 

\begin{example}
Suppose $k=\dF_q$ is a finite field of characteristic $p$. Then 
$G_k\simeq \widehat\dZ$ is topologically generated by a single element, called 
the Frobenius, and denoted $\frob_q$. Thus the representation $\rho_{A,\ell}$ 
is determined by the (conjugacy class of the) matrix $\rho_{A,\ell}(\frob_q)$. 
The map $\frob_q$ acts on each $A[\ell^n]$ by raising coordinates to the 
$q$th power. We also had an endomorphism $\Phi_A\in \End(A)$ that acts on 
$A[\ell^n]$ in the same way. In other words, 
$\rho_{A,\ell}(\frob_q) = \Phi_{A,\ast}$, so 
\[
  \det(t\cdot 1-\rho_{A,\ell}(\frob_q)) = \det(t\cdot 1-\Phi_{A,\ast},T_\ell(A)) \text{,}
\]
where we called the latter polynomial $P_A(t)$; this is an element of 
$\dZ[t]$ of degree $2 g$ which is independent of $\ell$. Honda-Tate theory 
tells us that $A$ can be recovered (up to isogeny) from $P_A$, and hence 
$\rho_{A,\ell}$. In other words, the functor 
$V_\ell:\mathsf{AbVar}_{\dF_q}\otimes\dQ \to \mathsf{Rep}_{\dQ_\ell}(G_{\dF_q})$ 
reflects isomorphisms (it is in fact fully faithful). 
\end{example}

Later, we'll look at an abelian variety $A$ over a number field $k$. The following 
are hard conjectures of Tate (theorems of Faltings now). 

$\rho_{A,\ell}$ is semisimple 

the map $\hom(A,B)\otimes\dQ \to \hom_{G_k}(V_\ell A,V_\ell B)$ is an isomorphism 

this has the consequence that if $A$ is an abelian variety over a number field 
$k$, then $A$ is determined up to isogeny by $\rho_{A,\ell}$. 





% notes on 11-26-2013

Let $A$ be an abelian variety of dimension $d\geqslant 1$ over a field $k$. Let 
$\ell$ be a prime invertible in $k$. We have groups 
$A[\ell^n]$ consisting of $\ell^n$-torsion in $A(\bar k)$. These patch together 
to give the Tate module 
\[
  T_\ell A = \varprojlim (A[\ell] \xleftarrow\ell A[\ell^2] \xleftarrow\ell A[\ell^3]\cdots) 
\]
Set $V_\ell A = T_\ell A\otimes\dQ$. Morally, 
\begin{align*}
  A[\ell^n] &= \h_1(A,\dZ/\ell^n) \\
  T_\ell A  &= \h_1(A,\dZ_\ell) \\
  V_\ell A &= \h_1(A,\dQ_\ell)
\end{align*}
The module $T_\ell A$ is free of rank $2 d$ over $\dZ_\ell$. Moreover, it has a 
natural action of $G_k=\gal(\bar k/k)$. In other words, we have a continuous 
homomorphism 
\[
  \rho_{A,\ell} : G_k \to \aut_{\dQ_\ell}(V_\ell A) \simeq \operatorname{GL}(2 d,\dQ_\ell) \text{.}
\]
It is natural to ask what $\rho_{A,\ell}$ ``knows'' about $A$, especially when 
$k$ is a number field. 

Recall that if $k=\dF_q$ is a finite field, then $G_{\dF_q}$ is topologically 
generated by the arithmetic Frobenius $\frob_q:x\mapsto x^q$. The matrix 
$\rho_{A,\ell}(\frob_q)$ has characteristic polynomial $P_A(t)\in \dZ[t]$, 
independent of $\ell$. Honda-Tate theory tells us that $P_A$ (and hence 
$\rho_{A,\ell}$ determines $A$ up to isogeny. 

(\textbf{mention Brauer-Nesbitt})

\begin{theorem}[Faltings' isogeny theorem]
Let $A$ and $B$ be abelian varieties over a number field $k$. For any prime 
$\ell$, we have $\rho_{A,\ell}\simeq \rho_{B,\ell}$ as $G_k$-modules if and 
only if $A$ and $B$ are isogenous over $k$. 
\end{theorem}

From this, we see that we can fruitfully study $A$ via $\rho_{A,\ell}$. For 
example, the rank of an abelian variety only depends on its isogeny class, so 
$\rank A$ only depends on $\rho_{A,\ell}$. 

(In both cases (finite or number field), the representation 
$\rho_{A,\ell}$, so $\rho_{A,\ell}(\frob_q)$ is diagonalizable over some 
field, so $\rho_{A,\ell}$ is determined by the characteristic polynomial of 
$\rho_{A,\ell}(\frob_q)$.)

Fix a number field $k$, and a finite place $v$ of $k$. Let 
$\fp\subset \fo=\fo_k$ be the corresponding maximal ideal. Let $k_v$ be the 
completion of $k$ at $v$. We choose $\bar k\subset \overline{k_v}$; this gives 
a map $G_{k_v} \to G_k$, defined by $\sigma\mapsto \sigma|_{\bar k}$. This map 
is only well-defined up to conjugation. In fact, this map is an injection. 
Reduction modulo $\fp$ gives a homomorphism 
$G_{k_v} \to G_{\kappa_v}=\widehat\dZ$, where $\kappa_v = \fo_v/\fp$ is the 
residue field of $\fp$. This map is surjective, so we have an exact sequence 
(where we write $D_v$ for the image of $G_{k_v}$ in $G_k$):
\[\xymatrix{
  1 \ar[r] 
    & I_v \ar[r] 
    & D_v \ar[r] 
    & \widehat\dZ \ar[r] 
    & 1
}\]
The group $G_{\kappa_v}$ is procyclic, with generator 
$\frob_{N v}$ where $N v = \# \kappa_v$. Write $\frob_v$ for a lift of 
$\frob_{N v}$ to $D_v$. The element $\frob_v\in G_k$ is only well-defined up to 
conjugacy and multiplication by $I_v$. 

As before, let $A$ be an abelian variety over $k$ with good reduction at $v$. 
In other words, there exists an abelian scheme $\cA$ over $\fo_v$ whose generic 
fiber is $A_{k_v}$. (As in a cartesian diagram:
\[\xymatrix{
  A_v \ar[r] \ar[d] 
    & \cA_v \ar[d] 
    & A_{k_v} \ar[l] \ar[d] \\
  \spec(\kappa_v) \ar[r] 
    & \spec(\fo_v) 
    & \spec(k_v) \ar[l] 
}\]
so that $A_v = \cA_{\kappa_v}$. In particular, $A_v$ is an abelian variety over 
$\kappa_v$, and we have a reduction map 
\[
  A(k_v) = \cA(\fo_v) \to \cA(\kappa_v) = A_v(\kappa_v) \text{.}
\]
Extending to algebraic closures, we get a map 
$A(\overline{k_v}) \to A_v(\overline{\kappa_v})$. This is a homomorphism with 
pro-$p$ kernel. Let $\ell\nmid N v$. At the level of torsion, we have 
isomorphisms 
\[
  A(\overline{k_v})[\ell^n] \to A_v(\overline{\kappa_v})[\ell^n]
\]
(the map is injective since its kernel is pro-$p$, and it's surjective by 
cardinality considerations: both groups have size $(\ell^n)^{2 d}$). This gives 
us an isomorphism 
$A(\bar k)[\ell^n] = A(\overline{k_v})[\ell^n]\xrightarrow\sim A_v(\kappa_v)[\ell^n]$. 
These groups have (compatible) actions of $G_k$, $G_{k_v}$ and $G_{\kappa_v}$. In 
particular, the inertia group $I_v$ acts trivially on $A(\bar k)[\ell^n]$. 

\textbf{(mention Ogg-N\'eron-Shafarevich)}

Summarizing: we have a well-defined action of $\frob_v$ on $A[\ell^n]$. 
Patching, $\frob_v$ has a well-defined action on $T_\ell A$. (This action 
matches the action of $\frob_{N v}$ on $T_\ell A_v$.) 

\begin{theorem}
Let $A$ be an abelian variety over $k$ with good reduction at $v$. Then 
\begin{enumerate}
  \item $\rho_{A,\ell}$ is unramified at $v$ (i.e. $\rho_{A,\ell}(I_v) = 1$) 
  \item Moreover, $\rho_{A,\ell}$ is well-defined up to conjugacy and has 
    characteristic polynomial $P_{A_v}(t)$ with integer coefficients that are 
    independent of $\ell$. 
\end{enumerate}
\end{theorem}

(\textbf{mention Serre's version using Haar measures})

Recall the \v Cebotarev density theorem. Let $L/k$ be a finite Galois extension 
of number fields. For $v$ unramified in $L/k$, there is a well-defined 
conjugacy class $[\frob_v]$ in $\gal(L/k)$. Take any conjugacy class $C$ in 
$\gal(L/k)$. Then 
\[
  \lim_{x \to \infty} \frac{\{\# v : N v \leqslant x \text{ and } [\frob_v]=C\}}{\# \{v:N v\leqslant x\}} = \frac{\# C}{\# \gal(L/k)}
\]
(In particular, each conjugacy class is the Frobenius for infinitely many 
primes.) 

For example, if $L=\dQ(\zeta_n)$, then $\gal(L/\dQ)$ is naturally isomorphic to 
$(\dZ/n)^\times$. For $p\nmid n$, one has $\frob_p \mapsto p\in (\dZ/n)^\times$. 
Dirichlet's theorem says that for $a\in (\dZ/n)^\times$, there exist infinitely 
many $p$ such that $p\equiv a\pmod n$. 

(\textbf{sort all this out})

\begin{theorem}[N\'eron-Ogg-Shafarevich]
Let $A$ be an abelian variety over a number field $k$. Let $v$ be a place of 
$k$, and let $\ell$ be a prime with $v\nmid \ell$. Then $A$ has good reduction 
at $v$ if and only if $\rho_{A,\ell}$ is unramified at $v$. 
\end{theorem}
\begin{proof}
Find Serre-Tate paper.
\end{proof}

Let's define the $L$-function of a general abelian variety. For a place $v$, 
choose a prime $\ell$ with $v\nmid \ell$. The action of 
$\frob_v$ on $T_\ell A$ is only well-defined up to the action of 
$I_v$. So $(T_\ell A)^{I_v}$ has a well-defined action of $\frob_v$. Set 
\[
  L_v(A,t) = \det\left(1-\rho_{A,\ell}(\frob_v)\cdot t,(T_\ell A)^{I_v}\right) \text{.}
\]
(This is the revere of $P_{A_v}(t)$ if $A$ has good reduction at $v$.) We define 
\[
  L(A,s) = \prod_{v\nmid \infty} L_v(A,(N v)^{-s}) \text{.}
\]

If $A=E$ is an elliptic curve over $\dQ$, then 
\[
  \rank_{\dZ_\ell} (T_\ell E)^{I_p}
  =
  \begin{cases}
    2 & \text{good reduction} \\
    1 & \text{mult. reduction} \\
    0 & \text{add. reduction}
  \end{cases}
\]

\textbf{(work on compatibility between this and classical definition of 
local factors at bad places.)}

The function $L(A,s)$ should have an analytic continuation, functional equation, 
it should satisfy BSD ($\ord_{s=1} L(A,s) = \rank_\dZ A(\dQ)$), and a ``fancy BSD'' 
with a precise prediction of the coefficient in Taylor series. 

The function $L(A,s)$ converges on some region $\{\Im s>c\}$. 









