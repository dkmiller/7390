% !TEX root = 7390-notes.tex





\section{Some theorems of Faltings}




For the rest of the semester, we'll talk about various results of Faltings. Let 
$A$ be an abelian variety of dimension $g\geqslant 1$ over a field $k$. As 
usual, $\bar k$ is the algebraic closure of $k$ and 
$G_k=\gal(\bar k/k)$ is the absolute Galois group. Let $\ell$ be a prime 
invertible in $k$. The groups $A[\ell^n]$ are the $\ell^n$-torsion subgroups 
of $A(\bar k)$. As an abstract group, $A[\ell^n]\simeq (\dZ/\ell^n)^{2 g}$. We 
have an inverse system 
\[\xymatrix{
  \cdots \ar[r] 
    & A[\ell^3] \ar[r]^-{\ell} 
    & A[\ell^2] \ar[r]^-\ell 
    & A[\ell] \text{.}
}\]
The $\ell$-adic \emph{Tate module} of $A$ is the inverse limit 
\[
  T_\ell A = \varprojlim_n A[\ell^n] = \left\{(x_n)_n : x_n\in A[\ell^n]\text{ and }\ell x_{n+1} = x_n\right\} \text{.}
\]
As an abstract group, $T_\ell A\simeq \dZ_\ell^{\oplus 2g}$. We set 
$V_\ell A = T_\ell A\otimes_\dZ \dQ$; this is a $2g$-dimensional 
$\dQ_\ell$-vector space. We should think of $V_\ell A$ as an algebraic 
analogue of $\h_1(A,\dQ_\ell)$. The group $G_k$ acts compatibly on the 
$A[\ell^n]$, hence it acts on $T_\ell A$. This action is continuous and 
respects the group structure on $T_\ell A$. Thus we have a continuous 
representation 
$\rho_{A,\ell} : G_k \to \aut_{\dQ_\ell}(T_\ell(A))\simeq GL(2 g,\dZ_\ell)$. 
(This actually lands inside $\operatorname{GSp}(2 g,\dZ_\ell)$, via the Weil 
pairing. \textbf{add details}). 

\begin{example}
Suppose $k=\dF_q$ is a finite field of characteristic $p$. Then 
$G_k\simeq \widehat\dZ$ is topologically generated by a single element, called 
the Frobenius, and denoted $\frob_q$. Thus the representation $\rho_{A,\ell}$ 
is determined by the (conjugacy class of the) matrix $\rho_{A,\ell}(\frob_q)$. 
The map $\frob_q$ acts on each $A[\ell^n]$ by raising coordinates to the 
$q$th power. We also had an endomorphism $\Phi_A\in \End(A)$ that acts on 
$A[\ell^n]$ in the same way. In other words, 
$\rho_{A,\ell}(\frob_q) = \Phi_{A,\ast}$, so 
\[
  \det(t\cdot 1-\rho_{A,\ell}(\frob_q)) = \det(t\cdot 1-\Phi_{A,\ast},T_\ell(A)) \text{,}
\]
where we called the latter polynomial $P_A(t)$; this is an element of 
$\dZ[t]$ of degree $2 g$ which is independent of $\ell$. Honda-Tate theory 
tells us that $A$ can be recovered (up to isogeny) from $P_A$, and hence 
$\rho_{A,\ell}$. In other words, the functor 
$V_\ell:\mathsf{AbVar}_{\dF_q}\otimes\dQ \to \mathsf{Rep}_{\dQ_\ell}(G_{\dF_q})$ 
reflects isomorphisms (it is in fact fully faithful). 
\end{example}

Later, we'll look at an abelian variety $A$ over a number field $k$. The following 
are hard conjectures of Tate (theorems of Faltings now). 

$\rho_{A,\ell}$ is semisimple 

the map $\hom(A,B)\otimes\dQ \to \hom_{G_k}(V_\ell A,V_\ell B)$ is an isomorphism 

this has the consequence that if $A$ is an abelian variety over a number field 
$k$, then $A$ is determined up to isogeny by $\rho_{A,\ell}$. 

