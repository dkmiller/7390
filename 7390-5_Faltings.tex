% !TEX root = 7390-notes.tex





\section{Some theorems of Faltings}





\subsection{Background and Tate's conjecture}

The goal of this section is to describe the relationships between a web of 
conjectures that Faltings proved in his groundbreaking paper 
\cite{fa86}. 

Let $A$ be a $d$-dimensional abelian variety over a field $k$. As usual, we 
write $\bar k$ for the algebraic closure of $k$ and $G_k=\gal(\bar k/k)$ for 
the absolute Galois group of $k$. Fix a prime $\ell$ invertible in $k$. The 
groups $A[\ell^n] = \{x\in A(\bar k):\ell^n x=0\}$ are abstractly isomorphic 
to $(\dZ/\ell^n)^{\oplus 2d}$, and carry a continuous action of $G_k$. They 
fit into an inverse system 
\[\xymatrix{
  A[\ell] 
    & A[\ell^2] \ar[l]_-\ell 
    & A[\ell^3] \ar[l]_-\ell 
    & \cdots \ar[l]
}\]
Put 
\[
  T_\ell A = \varprojlim A[\ell^n] = \left\{(x_n)\in \prod A[\ell^n] : \ell x_{n+1} = x_n\right\} \text{.}
\]
This is the \emph{$\ell$-adic Tate module of $A$}. As a $\dZ_\ell$-module, 
$T_\ell A \simeq \dZ_\ell^{\oplus 2 d}$. What makes $T_\ell A$ interesting is 
that it carries a continuous action of $G_k$, induced by the action of $G_k$ 
on the $A[\ell^n]$. In other words, after choosing a basis of $T_\ell A$, we 
have a continuous representation 
\[
  \rho_{A,\ell} : G_k \to \operatorname{GL}(2 d,\dZ_\ell) \text{.}
\]

The action of $G_k$ on $V_\ell = T_\ell A\otimes\dZ$ factors through 
the smaller group $\operatorname{GSp}(2 n,\dQ_\ell)$. One sees this via the 
\emph{Weil pairing}. Recall that $A^\vee = \picard_A^\circ$, which represents 
the functor 
\[
  T \mapsto \picard^\circ(A_T)/\picard(T) \text{.}
\]
See Section \ref{sec:picard-scheme} for more details. The identity morphism 
in $A^\vee(A^\vee)$ corresponds to an invertible sheaf $\sP$ on 
$A\times A^\vee$, called the \emph{Poincar\'e bundle}. This gives a 
biextension of $(A,A^\vee)$ by $\dG_m$ \cite[VII 2.9.5]{gr72}, and thus, by 
\cite[VIII 2.2]{gr72}, a $G_k$-equivariant pairing 
\[
  e:V_\ell A \times V_\ell A^\vee \to V_\ell \dG_m = \dQ_\ell(1) \text{.}
\]
See \cite[IX 1.0]{gr72} for a brief summary. Choosing a polarization 
$\lambda:A \to A^\vee$, we get an alternating nondegenerate pairing on 
$V_\ell A$, so after a suitable change of basis, the image of 
$\rho_{A,\ell}$ lands inside $\operatorname{GSp}(2 d,\dQ_\ell)$. We will mostly 
just think of $\rho_{A,\ell}$ as a representation with values in 
$\operatorname{GL}(2 d,\dZ_\ell)$ or $\operatorname{GL}(2 d,\dQ_\ell)$. It is 
natural to ask how much $\rho_{A,\ell}$ ``knows about'' $A$, especially if $k$ 
is a number field, or more generally, a finitely generated field. 

Let $X$ be a nice variety over a finitely generated field $k$. For each $i$, 
there is a canonical homomorphism 
\[
  \operatorname{cl}:\chow^i(X_{\bar k}) \to \h_\text{\'et}^{2 i}\left(X_{\bar k},\dQ_\ell\right)\text{,}
\]
defined in \cite[VI 2.2.10]{de77}. One calls $\operatorname{cl}(Z)$ the 
cohomology class associated with a cycle $Z$. 

\begin{conjecture}[Tate]
The subset of $\h^{2 i}(X_{\bar k},\dQ_\ell)(i)$ consisting of elements with 
open stabilizer in $G_k$ is generated by $\operatorname{cl}(A^i(X_{\bar k}))$. 
\end{conjecture}

This is Conjecture 1 in \cite{ta65}. Often when one speaks of the ``Tate 
conjecture'' one means the following special case. 

\begin{conjecture}[Tate]
Let $A,B$ be abelian varieties over a finitely generated field $k$. For any 
prime $\ell$ invertible in $k$, the natural map 
\[
  \hom_k(A,B)\otimes\dQ_\ell \to \hom_{G_k}(V_\ell A, V_\ell B) \text{,}
\]
is a bijection. 
\end{conjecture}

See the remarks after Conjecture 1 in Tate's paper for a proof that the second 
conjecture follows from the first. Another way of stating the (second version 
of the) Tate conjecture is that for any finitely generated field $k$, the 
functor $V_\ell:\mathsf{AbVar}_k^\text{iso} \to \mathsf{Rep}_{G_k}(\dQ_\ell)$ 
is fully faithful. 

\begin{example}
Let $k=\dF_q$ be a finite field. Then $G_k$ is naturall isomorphic to 
$\widehat\dZ$, the profinite completion of $\dZ$. Here $1\in \widehat\dZ$ 
corresponds to the \emph{arithmetic Frobenius} $\frob_q\in G_{\dF_q}$, given by 
$x\mapsto x^q$. Representations 
$\rho:G_{\dF_q} \to \operatorname{GL}(n,\dQ_\ell)$ are determined by 
$\rho(\frob_q)$. If such a representation is semisimple, the 
\emph{Brauer-Nesbitt theorem} tells us that $\rho$ is determined by the 
characteristic polynomial of $\rho(\frob_q)$. For an abelian variety $A$ over 
$\dF_q$, we know that the characteristic polynomial of 
$\rho_{A,\ell}(\frob_q)$ is $P_A(t)\in \dZ[t]$, which determines $A$ up to 
isogeny by Honda-Tate theory. 
\end{example}

Since we will be using characteristic polynomials quite a lot, let us state a 
suitably general version of the Brauer-Nesbitt theorem. Fix a field $k$, and 
for an arbitrary group $G$, let $K_0(G)$ denote the Grothendieck group of 
finite-dimensional $k$-representations of $G$. By the ``characteristic 
polynomial'' of a representation $\rho:G \to \operatorname{GL}_k(V)$, we mean 
the formal power series 
\[
  \chi_\rho(g) = \frac{1}{\det(1-\rho(g)\cdot t, V)} \in \Lambda(k) = 1+t k\llbracket t\rrbracket \text{.}
\]

\begin{theorem}[Brauer-Nesbitt]
If $S$ spans $k[G]$ as a $k$-vector space, then the map 
$\chi:K_0(G) \to \Lambda(k)^S$ given by $[\rho]\mapsto \chi_\rho$ is an 
injection. 
\end{theorem}
\begin{proof}
This is Theorem 5.21 of \cite{eg11}. 
\end{proof}

\begin{corollary}
If $G$ is a group and $\rho_1,\rho_2:G \to \operatorname{GL}_k(V)$ are two 
semisimple representations with equal characteristic polynomials, then 
$\rho_1\simeq \rho_2$. 
\end{corollary}






% notes on 11-26-2013

\begin{theorem}[Faltings' isogeny theorem]
Let $A$ and $B$ be abelian varieties over a number field $k$. For any prime 
$\ell$, we have $\rho_{A,\ell}\simeq \rho_{B,\ell}$ as $G_k$-modules if and 
only if $A$ and $B$ are isogenous over $k$. 
\end{theorem}

From this, we see that we can fruitfully study $A$ via $\rho_{A,\ell}$. For 
example, the rank of an abelian variety only depends on its isogeny class, so 
$\rank A$ only depends on $\rho_{A,\ell}$. 

If $k$ is either finite or a global field, the representation $\rho_{A,\ell}$ 
is semisimple, so $\rho_{A,\ell}$ is determined by the characteristic 
polynomial of $\rho_{A,\ell}(\frob_q)$. For this, one needs the 
\emph{\v Cebotarev density theorem}. 

\textbf{(add details here or later)}





\subsection{Image of Frobenius for number fields}

Fix a number field $k$, and a finite place $v$ of $k$. Let 
$\fp\subset \fo=\fo_k$ be the corresponding maximal ideal. Let $k_v$ be the 
completion of $k$ at $v$. We choose $\bar k\subset \overline{k_v}$; this gives 
a map $G_{k_v} \to G_k$, defined by $\sigma\mapsto \sigma|_{\bar k}$. This map 
is only well-defined up to conjugation. In fact, this map is an injection. 
Reduction modulo $\fp$ gives a homomorphism 
$G_{k_v} \to G_{\kappa_v}=\widehat\dZ$, where $\kappa_v = \fo_v/\fp$ is the 
residue field of $\fp$. This map is surjective, so we have an exact sequence 
(where we write $D_v$ for the image of $G_{k_v}$ in $G_k$):
\[\xymatrix{
  1 \ar[r] 
    & I_v \ar[r] 
    & D_v \ar[r] 
    & \widehat\dZ \ar[r] 
    & 1
}\]
The group $G_{\kappa_v}$ is procyclic, with generator 
$\frob_{N v}$ where $N v = \# \kappa_v$. Write $\frob_v$ for a lift of 
$\frob_{N v}$ to $D_v$. The element $\frob_v\in G_k$ is only well-defined up to 
conjugacy and multiplication by $I_v$. 

As before, let $A$ be an abelian variety over $k$ with good reduction at $v$. 
In other words, there exists an abelian scheme $\cA$ over $\fo_v$ whose generic 
fiber is $A_{k_v}$. (As in a cartesian diagram:
\[\xymatrix{
  A_v \ar[r] \ar[d] 
    & \cA_v \ar[d] 
    & A_{k_v} \ar[l] \ar[d] \\
  \spec(\kappa_v) \ar[r] 
    & \spec(\fo_v) 
    & \spec(k_v) \ar[l] 
}\]
so that $A_v = \cA_{\kappa_v}$. In particular, $A_v$ is an abelian variety over 
$\kappa_v$, and we have a reduction map 
\[
  A(k_v) = \cA(\fo_v) \to \cA(\kappa_v) = A_v(\kappa_v) \text{.}
\]
Extending to algebraic closures, we get a map 
$A(\overline{k_v}) \to A_v(\overline{\kappa_v})$. This is a homomorphism with 
pro-$p$ kernel. Let $\ell\nmid N v$. At the level of torsion, we have 
isomorphisms 
\[
  A(\overline{k_v})[\ell^n] \to A_v(\overline{\kappa_v})[\ell^n]
\]
(the map is injective since its kernel is pro-$p$, and it's surjective by 
cardinality considerations: both groups have size $(\ell^n)^{2 d}$). This gives 
us an isomorphism 
$A(\bar k)[\ell^n] = A(\overline{k_v})[\ell^n]\xrightarrow\sim A_v(\kappa_v)[\ell^n]$. 
These groups have (compatible) actions of $G_k$, $G_{k_v}$ and $G_{\kappa_v}$. In 
particular, the inertia group $I_v$ acts trivially on $A(\bar k)[\ell^n]$. 

\textbf{(mention Ogg-N\'eron-Shafarevich)}

Summarizing: we have a well-defined action of $\frob_v$ on $A[\ell^n]$. 
Patching, $\frob_v$ has a well-defined action on $T_\ell A$. (This action 
matches the action of $\frob_{N v}$ on $T_\ell A_v$.) 

\begin{theorem}
Let $A$ be an abelian variety over $k$ with good reduction at $v$. Then 
\begin{enumerate}
  \item $\rho_{A,\ell}$ is unramified at $v$ (i.e. $\rho_{A,\ell}(I_v) = 1$) 
  \item Moreover, $\rho_{A,\ell}$ is well-defined up to conjugacy and has 
    characteristic polynomial $P_{A_v}(t)$ with integer coefficients that are 
    independent of $\ell$. 
\end{enumerate}
\end{theorem}

(\textbf{mention Serre's version using Haar measures})

Recall the \v Cebotarev density theorem. Let $L/k$ be a finite Galois extension 
of number fields. For $v$ unramified in $L/k$, there is a well-defined 
conjugacy class $[\frob_v]$ in $\gal(L/k)$. Take any conjugacy class $C$ in 
$\gal(L/k)$. Then 
\[
  \lim_{x \to \infty} \frac{\{\# v : N v \leqslant x \text{ and } [\frob_v]=C\}}{\# \{v:N v\leqslant x\}} = \frac{\# C}{\# \gal(L/k)}
\]
(In particular, each conjugacy class is the Frobenius for infinitely many 
primes.) 

For example, if $L=\dQ(\zeta_n)$, then $\gal(L/\dQ)$ is naturally isomorphic to 
$(\dZ/n)^\times$. For $p\nmid n$, one has $\frob_p \mapsto p\in (\dZ/n)^\times$. 
Dirichlet's theorem says that for $a\in (\dZ/n)^\times$, there exist infinitely 
many $p$ such that $p\equiv a\pmod n$. 

(\textbf{sort all this out})

\begin{theorem}[N\'eron-Ogg-Shafarevich]
Let $A$ be an abelian variety over a number field $k$. Let $v$ be a place of 
$k$, and let $\ell$ be a prime with $v\nmid \ell$. Then $A$ has good reduction 
at $v$ if and only if $\rho_{A,\ell}$ is unramified at $v$. 
\end{theorem}
\begin{proof}
Find Serre-Tate paper.
\end{proof}

Let's define the $L$-function of a general abelian variety. For a place $v$, 
choose a prime $\ell$ with $v\nmid \ell$. The action of 
$\frob_v$ on $T_\ell A$ is only well-defined up to the action of 
$I_v$. So $(T_\ell A)^{I_v}$ has a well-defined action of $\frob_v$. Set 
\[
  L_v(A,t) = \det\left(1-\rho_{A,\ell}(\frob_v)\cdot t,(T_\ell A)^{I_v}\right) \text{.}
\]
(This is the revere of $P_{A_v}(t)$ if $A$ has good reduction at $v$.) We define 
\[
  L(A,s) = \prod_{v\nmid \infty} L_v(A,(N v)^{-s}) \text{.}
\]

If $A=E$ is an elliptic curve over $\dQ$, then 
\[
  \rank_{\dZ_\ell} (T_\ell E)^{I_p}
  =
  \begin{cases}
    2 & \text{good reduction} \\
    1 & \text{mult. reduction} \\
    0 & \text{add. reduction}
  \end{cases}
\]

\textbf{(work on compatibility between this and classical definition of 
local factors at bad places.)}

The function $L(A,s)$ should have an analytic continuation, functional equation, 
it should satisfy BSD ($\ord_{s=1} L(A,s) = \rank_\dZ A(\dQ)$), and a ``fancy BSD'' 
with a precise prediction of the coefficient in Taylor series. 

The function $L(A,s)$ converges on some region $\{\Im s>c\}$. 









