% !TEX root = 7390-notes.tex










% notes on 09-12-2013
\section{The Mordell-Weil theorem}





\subsection{Statement and generalizations}

The goal of this section is to give an essentially complete proof of the 
\emph{Mordell-Weil theorem}. Throughout the section, $k$ will denote a 
field, often a \emph{number field} (finite field extension of $\dQ$). 
Important examples are the \emph{quadratic fields} $k=\dQ(\sqrt d)$ and 
\emph{cyclotomic fields} $\dQ(\zeta_n)$. 

\begin{theorem}[Mordell-Weil]
Let $A$ be an abelian variety over a number field $k$. Then the abelian group 
$A(k)$ is finitely generated.
\end{theorem}

This is clearly false if $k=\dC$ and $A\ne 0$, for then $A(\dC)$ is a complex Lie 
group, hence uncountable. In fact, whenever $k$ is a local field, $A(k)$ is a 
Lie group over $k$, hence uncountable. The Mordell-Weil theorem does holds 
whenever $k$ is finitely generated over its prime field. In this case, basic algebra shows 
that $A(k) = A(k)_\text{tors}\oplus \dZ x_1\oplus \cdots \oplus \dZ x_r$, where 
the $x_i$ are linearly independent over $\dZ$. We call $\rank A=r$ 
the \emph{rank} of $A$. Mordell proved the theorem for $A$ an elliptic 
curve over $\dQ$, demonstrating an assertion of Poincar\'e.
%Mordell 1922, Poincare 1901

\begin{example}
Let $E\subset \dP_\dQ^2$ be the projective closure of the affine curve 
defined by $y=x^3+2 x+3$, with $O=(0:1:0)$ the point at infinity. The curve 
has a group law such that $a+b+c=0$ if and only if $a,b,c$ are colinear. 
Alternatively, let $J=\jac E$, The point $O\in E(\dQ)$ induces an embedding 
$E\hookrightarrow J$ sending $O$ to $0$. This is an isomorphism, and we can 
use it to transfer the group structure of 
$J$ to $E$. The curve $E$ has an obvious rational point $(-1,0)$ of order 
two. Another rational points is $(3,6)$. Their sum is 
$\left(\frac 1 4,-\frac{15}{16}\right)$. One can show that 
$E(\dQ) = \langle (-1,0)\rangle \oplus \langle (3,6)\rangle$, where 
$(-1,0)$ has order two and $(3,6)$ has infinite order. So 
$E(\dQ) = \dZ/2\oplus \dZ$. 
\end{example}

\begin{example}
Let $E/\dQ$ be the projective closure of $y^2+y = x^3+x^2 - 2 x$. We claim 
that $E(\dQ) = \langle (0,0),(1,0)\rangle\simeq \dZ^{\oplus 2}$. As an 
exercise, try to find ten more points in $E(\dQ)$. 
\end{example}

A result that motivated Weil is the following conjecture of Mordell (now a 
theorem of Faltings). 

\begin{theorem}[Faltings]
If $C$ is a nice curve over a number field $k$ with genus $g\geqslant 2$, then 
$C(k)$ is finite.
\end{theorem}

Mordell's conjecture fails if $g\leqslant 1$. For $g=0$, 
$\dP^1$ has lots of rational points, and we have seen examples of 
elliptic curves with infinitely many rational points. Here is a heuristic. 
Assume $C(k)\ne\varnothing$ and consider the canonical embedding 
$C\hookrightarrow J$. We have $C(k)=C\cap J(k)$. The set $C$ has positive 
codimension in $J$, and $J(k)$ is a finitely generated abelian group. So 
$C(k)$ is the intersection of two ``sparse'' subsets of $J$. One would expect 
this forces $C(k)$ to be small. This heuristic is validated by the following 
theorem, originally known as the \emph{Mordell-Lang conjecture}.

\begin{theorem}[Faltings]
Let $A$ be an abelian variety over an algebraically closed field $k$ of 
characteristic zero, and let 
$\Gamma$ be a finitely generated subgroup of $A(k)$. If $X\subset A$ is a 
subvariety, then there is a finite set $S\subset \Gamma$ and a finite set 
$\{B_s:s\in S\}$ of abelian subvarieties of $A$ such that 
\[
  X(k)\cap \Gamma = \bigcup_{s\in S} \left(s+ B_s(k)\cap \Gamma\right) \text{.}
\]
\end{theorem}
\begin{proof}
See \cite{mc95} for a proof in the case $k=\dC$. The general case follows by 
the Lefschetz principle. McQuillan actually proves the 
theorem for a broader class of group varieties than abelian varieties. 
\end{proof}

\begin{corollary}
Let $A$ be an abelian variety over $\dC$, and let $C$ be a nice curve in $A$ of 
genus $g\geqslant 2$. Let $\Gamma$ be a finitely generated subgroup of 
$A(\dC)$. Then $C(\dC)\cap \Gamma$ is finite. 
\end{corollary}
\begin{proof}
Since $C$ has genus $g\geqslant 2$, it cannot contain a nontrivial abelian 
variety. Thus each $B_s = 0$, so the theorem yields $C(\dC)\cap \Gamma = S$ 
for some finite set $S\subset \Gamma$. 
\end{proof}

There is a relative version of the Mordell-Lang conjecture known as the 
\emph{Lang-N\'eron theorem}. Let $K/k$ be a \emph{regular} field extension, 
that is, $\bar k\cap K = k$ and $K/k$ is separable. If $A$ is an abelian 
variety defined over $K$, then there is an abelian variety 
$\trace_{K/k}(A)$ defined over $k$ together which a morphism 
$\tau:\trace_{K/k}(A)_K\to A$ that is initial among abelian 
varieties $B/k$ with morphisms $B_K\to A$. One calls 
$\trace_{K/k}(A)$ the \emph{$K/k$-trace of $A$}. Intuitively, 
$\trace_{K/k}(A)$ is the smallest abelian subvariety of $A$ defined 
over $k$. A proof of the following theorem can be found in \cite{co06}. 

\begin{theorem}[Lang-N\'eron]
Let $K/k$ be a finitely generated regular extension, and let $A$ be an abelian 
variety over $K$. Then the group $A(K)/\trace_{K/k}(A)(k)$ is 
finitely generated.
\end{theorem}

This implies the Mordell-Weil theorem for finitely generated fields. If $K$ is 
a finitely generated field, let $k$ be the algebraic closure of the prime field 
of $K$ within $K$. Then $K/k$ is regular, and by the usual Mordell-Weil 
theorem, $\trace_{K/k}(A)(k)$ is finitely generated, so $A(K)$ has a 
finitely generated subgroup with finitely generated quotient. It follows that 
$A(K)$ is finitely generated.





\subsection{Plan of the proof}

Let $A$ be an abelian variety over a number field $k$. Our proof has three 
parts:
\begin{enumerate}
  \item construct a ``height function'' $|\cdot|:A(k)\to \dR_{\geqslant 0}$ 
    with good properties
  \item prove the weak Mordell-Weil theorem: $A(k)/n A(k)$ is finite for all 
    $n\geqslant 2$
  \item show that 1 and 2 formally imply the full Mordell-Weil theorem 
\end{enumerate}
We will prove 3 here, spend the next several sections on 2, and finally 
construct a good height function in \ref{sec:neron-tate}. For the 
sake of space, write $A(k)/n$ instead of $A(k)/n A(k)$, and 
$\torsion n A(k)$ instead of $\{x\in A(k):n\cdot x=0\}$. 

\begin{lemma}
Let $A$ be an abelian group with a function 
$|\cdot|:A\to \dR_{\geqslant 0}$ such that for some $n\geqslant 2$, 
\begin{itemize}
  \item for all $c>0$, the set $B_c = \{x\in A:|x|\leqslant c\}$ 
    is finite
  \item $|x-y|\leqslant |x|+|y|$ and $|n x|=n |x|$ for all 
    $x,y\in A$, $n\in \dN$
  \item the group $A/n$ is finite
\end{itemize}
Then $A$ is finitely generated. 
\end{lemma}
\begin{proof}
Let $\{a_i\}$ be a (finite) set 
of coset representatives for $A/n$. Let $c=2\sup \{|a_i|\}$. We claim that $A$ 
is generated by the finite set $B_c$. This will be shown ``by descent.'' Fix 
$n\geqslant 2$ and let $x_1\in A$. By our assumptions, we can write 
$x_1 = y_{i_1} + n x_2$ with $\{a_i\}$, and one see that 
\[
  |x_2| = \frac 1 n |x_1 - y_{i_1}| 
    \leqslant \frac{|x_1| + |y_{i_1}|}{n} 
    \leqslant \frac 1 n |x_1| + \frac{c}{2 n} \text{.}
\]
Setting $x_{r+1} = y_{i_r} + n x_r$ with $y_{i_r}\in \{a_i\}$, we can continue 
this process, obtaining the inequality 
\[
  |x_{r+1}| \leqslant \frac 1 n |x_r| + \frac{c}{2 n}
    \leqslant \frac{1}{n^r} |x_1| + \frac 1 2\left(\frac 1 n + \cdots + \frac{1}{n^r}\right) c 
    \leqslant \frac{|x_1|}{n^r} + \frac c 2 \text{.}
\]
For $r\gg 0$, the quantity on the right is less than $c$. Thus $x_1$ is in the 
subgroup of $A$ generated by $B_c$ and $\{a_i\}$. Since $|a_i|\leqslant c$ for 
each $i$, we have $A=\langle B_c\rangle$. 
\end{proof}

To prove the weak Mordell-Weil theorem, we will use group cohomology 
extensively. For some motivation, let $A$ be an abelian variety over a field 
$k$ of characteristic zero. 
Let $\torsion n A$ be the kernel of $\cdot n : A(\bar k)\to A(\bar k)$. We can take 
$G_k$-invariants of $\torsion n A$, resulting in an exact sequence: 
\[\xymatrix{
  0 \ar[r] 
    & (\torsion n A)^{G_k} \ar[r] 
    & A(k) \ar[r]^-n 
    & A(k) \text{.}
}\]
We are interested in continuing this exact sequence to the right, i.e. in 
constructing a long exact sequence 
\[
\begin{tikzcd}
  0 \rar 
    & (\torsion n A)^{G_k} \rar 
    & A(k) \rar 
    & A(k) \ar[out=355,in=175, overlay]{dll} \\
  & \h^1(G_k,\torsion n A) \rar 
    & \h^1\left(G_k,A(\bar k)\right) \rar 
    & \h^1\left(G_k,A(\bar k)\right) \rar
    & \cdots
\end{tikzcd}
\]
This fits into a general framework of derived functors, but we will mostly 
just use the concrete definition of $\h^1(G_k,-)$ using cocycles and 
coboundaries. 

Here is a very brief explanation of the construction of $|\cdot|$. Here we 
require $k$ to be a number field. There is a height function on 
$\dP^n(\bar k)$ defined by  
\[
  h(x_0:\cdots:x_n) = \frac{1}{[L:\dQ]} \sum_v \log \sup\{|x_i|_v\} \text{,}
\]
where $L/k$ is some extension containing the $x_i$ and $|\cdot|_v$ is the 
normalized absolute value associated with $v$. For a very ample divisor $c$ on 
$A$, we get an embedding $\varphi_c:A\to \dP^n$, and thus a height function 
$h_c:A(\bar k)\to \dR_{\geqslant 0}$. As it stands, this is not uniquely 
defined, 
but there is a unique way of adjusting $h_c$ by a bounded function on 
$A(\bar k)$ to get a function $\widehat h_c:A(\bar k)\to \dR$ such that 
\[
  (x,y)\mapsto \langle x,y\rangle_c = \frac 1 2 \left(\widehat h_c(x+y) - \widehat h_c(x) - \widehat h_c(y) \right)
\]
is bilinear. One calls $\widehat h_c$ the \emph{N\'eron-Tate height} associated 
with $c$. The function $|\cdot|$ in the proof can be taken to be 
$|\cdot|_c=\widehat h_c^{1/2}$ for any very ample even divisor $c$.





Our proof of the Mordell-Weil theorem is very nearly effective. Given a set of 
generators for 
$A(k)/n$, it gives an algorithm for finding a set of generators for 
$A(k)$. Moreover, one can choose any integer $n$. Most people use $n = 2$ when doing 
computations. 

%In the next section, we will give a proof of the weak 
%Mordell-Weil theorem using Galois cohomology. Let $\bar k$ be the algebraic 
%closure of $k$, and let  $G_k = \gal(\bar k/k)$ be the absolute Galois group of 
%$k$. The abelian group $A(\bar k)$ has a continuous $G_k$-action. One way to 
%see this is to embed $A$ into some $\dP^N$, and let $G_k$ act on 
%coordinates. A fancier way to see this is to note that 
%$A(\bar k) = \hom(\spec(\bar k),A)$, so if $\sigma\in G_k$, $x\in A(\bar k)$, 
%the point $^\sigma x$ is the composite 
%$\spec(\bar k)\xrightarrow{\sigma^*} \spec(\bar k) \xrightarrow{x} A$. 










% lecture 09-17-2013
\subsection{Group cohomology}

Let $k$ be a number field, $\bar k$ 
an algebraic closure of $k$, and $G_k=\gal(\bar k/k)$ the \emph{absolute Galois 
group} of $k$. This is a \emph{profinite group} (compact, totally connected 
Hausdorff group) with a basis of neighborhoods of $1$ being the groups 
$G_L=\gal(\bar k/L)$ where $L$ ranges over the finite extensions of $k$. Let 
$A$ be an abelian variety over $k$ of dimension $d\geqslant 1$. There is no 
harm in thinking of $k=\dQ$ and $A$ as an elliptic curve. 
The abelian group $A(\bar k)$ naturally has a continuous $G_k$-action. One way 
to see this is to embed $A$ into some huge projective space and let $G_k$ act 
on each coordinate. A ``fancier'' way to see this is to note that 
$A(\bar k) = \hom(\spec(\bar k),A)$, so if $\sigma\in G_k$, $\in A(\bar k)$, 
the point $\sigma(x)$ is the composite 
\[\xymatrix{
  \spec(\bar k) \ar[r]^-{\sigma^\ast} \ar[dr]_-{\sigma(x)^\ast} 
    & \spec(\bar k) \ar[d]^-x \\
  & A 
}\]

The homomorphism $A\xrightarrow n A$ that sends $x$ to $n\cdot x$ is an 
isogeny. Let $\torsion n A$ be the $n$-torsion subgroup of $A(\bar k)$, i.e. 
\[
  \torsion n A = \{x\in A(\bar k) : n\cdot x = 0\}
\]
If $k$ were a field of positive characteristic $p$ and $p\mid n$, it would be 
better to think of $A[n]$ as the scheme $A\times_A 0$ via $n:A\to A$. In any 
case, there is an exact sequence 
\[\xymatrix{
  0 \ar[r] 
    & \torsion n A \ar[r] 
    & A(\bar k) \ar[r]^-n 
    & A(\bar k) \ar[r] 
    & 0 \text{.}
}\]
Recall that if $G$ is an arbitrary group acting on some abelian group $M$, we 
define the module of \emph{$G$-invariants} of $M$ by 
\[
  M^G=\{m\in M:\sigma m = m\text{ for all } \sigma\in G\} \text{.}
\]
Taking $G_k$-invariants of the above short exact sequence, we get an exact 
sequence 
\[\xymatrix{
  0 \ar[r] 
    & (\torsion n A)^{G_k} \ar[r] 
    & A(k) \ar[r]^-n 
    & A(k) \text{.}
}\]
We don't usually have exactness on the right because the functor $(-)^{G_k}$ is 
not right-exact. For example, if $k$ is a number field, the $n$-th power map 
$(-)^n:\bar k^\times \to \bar k^\times$ is surjective, but 
$(-)^n:(\bar k^\times)^{G_k} = k^\times \to k^\times$ is not. 

We can define $\h^\bullet(G_k,\torsion n A)$ using the formalism of derived functors. Consider the 
category $G_k\text{-}\mathsf{Mod}$ of (discrete) abelian groups with continuous 
$G_k$-action. This is an abelian category with enough injectives. The functor 
$\Gamma=(-)^{G_k}:G_k\text{-}\mathsf{Mod}\to \mathsf{Ab}$ is left-exact, we we 
define the group cohomology as the derived functors of $\Gamma$, i.e. 
$\h^\bullet(G_k,-) = \mathsf{R}^\bullet \Gamma$. In particular, 
$\h^0(G_k,M) = M^{G_k}$ for any discrete $G_k$-module $M$. 

Choose $x\in A(k)$. Then $x=n\cdot y$ for some $y\in A(\bar k)$. Take 
$\sigma\in G_k$. Then $x=\sigma x$, so $\sigma(n\cdot y) = n\cdot \sigma(y)$. 
Then $n\cdot (\sigma y - y) = n\cdot \sigma y - n\cdot y = 0$, so 
$\sigma y-y\in \torsion n A$. Thus we have a map (\emph{not} usually a homomorphism) 
$\varphi:G_k\to \torsion n A$, $\sigma\mapsto \sigma y - y$. Take $\sigma,\tau\in G_k$. 
Then one computes 
\begin{align*}
  \varphi(\sigma\tau) &= \sigma\tau(y) - y \\
    &= \sigma(\tau y - y) + \sigma y -y \\
    &= \sigma\varphi(\tau) + \varphi(\sigma) \text{.}
\end{align*}
So $\varphi$ is a homomorphism precisely when the action of $G_k$ on 
$\torsion n A$ is trivial. Moreover, there is a number field $L/k$ such that 
$G_L\subset G_k$ fixes $y$. In particular, $\varphi(G_L)=0$, i.e. the map 
$\varphi:G_k\to \torsion n A$ is continuous. Suppose we choose some $y'$ distinct from 
$y$ with $x=n\cdot y'$. We could define $\varphi':G_k\to \torsion n A$ by 
$\sigma\mapsto \sigma y'-y'$. Since $n(y-y')=0$, we have $y'-y\in \torsion n A$, hence 
$y'=y+\alpha$ for some $\alpha\in \torsion n A$. We now have 
\begin{align*}
  \varphi'(\sigma) &= \sigma y'-y' \\
    &= \sigma(y+\alpha) - (y+\alpha) \\
    &= \sigma y - y + \sigma\alpha-\alpha \\
    &= \varphi(\sigma) + \sigma \alpha-\alpha \text{.}
\end{align*}
Maps $G_k\to \torsion n A$ of the form $\sigma\mapsto \sigma \alpha-\alpha$ will be 
called \emph{coboundaries}. 

Suppose we have another point $x'\in A(k)$. Choose a $y'$ with $x'=n y'$. Then 
$x+x'=n(y+y')$, and the point $x+x'$ gives rise to a map $G_k\to \torsion n A$, 
$\sigma\mapsto \sigma(y+y')-(y+y')$; this map is just $\varphi+\varphi'$, where 
$\varphi':\sigma\mapsto \sigma y'-y$. 

\begin{definition}
Let $G$ be a profinite group, $M$ a discrete $G$-module. The group $Z^1(G,M)$ 
of \emph{$1$-cocycles} consists of continuous maps $\varphi:G\to M$ such that 
$\varphi(\sigma\tau)=\sigma\varphi(\tau)+\varphi(\sigma)$. The group $B^1(G,M)$ 
of \emph{$1$-coboundaries} consists of continuous maps $\varphi:G\to M$ of the 
form $\sigma\mapsto \sigma\alpha-\alpha$ for some $\alpha\in M$. The 
\emph{first cohomology of $G$ with coefficients in $M$} is 
\[
  \h^1(G,M) = Z^1(G,M) / B^1(G,M) \text{.}
\]
\end{definition}
One can prove using a canonical projective resolution of $M$ that this agrees 
with the derived functor definition. It is not hard to  check directly that if 
$0\to M'\xrightarrow f M\xrightarrow g M'' \to 0$ is an exact sequence of 
$G_k$-modules, then there is a natural exact sequence:
\begin{center}
\begin{tikzcd}
  0 \rar 
    & {M'}^G \rar{f}
    & M^G \rar{g}
    & {M''}^G \ar[out=355,in=175, overlay]{dll}{\delta} \\
  & \h^1(G,M') \rar{f_\ast}
    &\h^1(G,M) \rar{g_\ast} 
    & \h^1(G,M'') \text{.}
\end{tikzcd}
\end{center}
where $\delta(x)$ is defined as follows. Choose a lift $\tilde x$ in 
$M$ of $x$, and let $\delta(x)(\sigma) = f^{-1}(\sigma \tilde x - \tilde x)$. One can 
check that this is a cocycle, and that choosing a different $\tilde x$ changes 
$\delta(x)(\sigma)$ by a coboundary. 

\hard{
If $G$ is a discrete group, then the cohomology $\h^\bullet(G,M)$ can be 
interpreted as an $\operatorname{Ext}$-group. It is easy to see that the 
category of $G$-module is equivalent to the category of $\dZ[G]$-modules, and 
that $M^G\simeq \hom_{\dZ[G]}(\dZ,G)$. It follows that 
$\h^\bullet(G,M)\simeq \operatorname{Ext}_{\dZ[G]}^\bullet(\dZ,M)$. If $G$ is a 
profinite group, then $\dZ[G]$ is the wrong ring to use. Instead, one considers 
the \emph{completed group ring} 
\[
  \dZ\llbracket G\rrbracket = \varprojlim_{\substack{N\triangleleft G \\ N\text{ open}}} \dZ[G/N]
\]
where we give each $\dZ[G/N]$ the discrete topology and 
$\dZ\llbracket G\rrbracket$ the inverse limit topology. The category of 
discrete $G$-modules with continuous action is equivalent to the category of 
discrete $\dZ\llbracket G\rrbracket$-modules with continuous action, 
$M^G\simeq \hom_{\dZ\llbracket G\rrbracket}(\dZ,M)$, whence 
$\h^\bullet(G,M) \simeq \operatorname{Ext}_{\dZ\llbracket G\rrbracket}^\bullet(\dZ,M)$. 

In the case that $G=G_k$, one can interpret the groups $\h^\bullet(G,M)$ as a 
special case of \'etale cohomology. Recall that a morphism $f:X\to S$ of 
schemes is \emph{\'etale} if it is flat and unramified. The \emph{\'etale 
site} of $S$, denoted $\et S$, is the full subcategory of $\mathsf{Sch}_S$ 
consisting of $X\to S$ that are \'etale. A collection $\{f_i:U_i\to X\}$ in 
$\et S$ is a \emph{cover} if the images of the $f_i$ cover $X$. With this 
notion of a cover, $\et S$ is a (subcanonical) site, so we can talk about 
sheaves and cohomology on $\et S$. The main example is: 
}

\begin{example}[$\star$]
Let $k$ be a field, and write $\et k$ for $\et{\spec(k)}$. Then the objects of 
$\et k$ are all 
of the form $\spec(L_1)\sqcup \cdots \sqcup \spec(L_n)\to \spec(k)$ for a 
finite family of separable field extensions $L_1,\dots,L_n$ of $k$. One can 
check that the category $\sheaf(\et k)$ of abelian sheaves on $\et k$ is 
equivalent to the category of discrete $G_k$-modules, via the functor 
$\sF\mapsto \sF_{\bar k} = \varinjlim_{L/k} \sF(\spec L)$, where $L$ ranves 
over all finite Galois extensions of $k$. If $M$ is a $G_k$-module, then there 
is a corresponding \'etale sheaf $\widetilde M$, determined by 
$\widetilde M(\spec L) = M^{G_L}$. Since 
$M^G\simeq \h^0(\et k,\widetilde M)$, we obtain that 
$\h^\bullet(G_k,M) \simeq \h^\bullet(\et k, \widetilde M)$. For more details, 
see \cite[I 2.4]{de77}. 
\end{example}

In our case, from the long exact sequence associated to 
$0\to \torsion n A\to A(\bar k) \xrightarrow n A(\bar k) \to 0$ yields a short 
exact sequence 
\[\xymatrix{
  0 \ar[r] 
    & A(k) / n \ar[r]^-\delta
    & \h^1\left(G_k,\torsion n A\right) \ar[r] 
    & \torsion{n}{\h^1\left(G_k,A(\bar k)\right)} \ar[r]
    & 0 \text{.}
}\]
We will try to prove that $A(k)/ n$ is finite by embedding it into a group 
we know is finite. Unfortunately $\h^1(G_k, \torsion n A)$ is infinite, so we cannot 
just use the above exact sequence to show that $A(k)/n$ is finite. 

In general, suppose $G$ is a (commutative) group scheme over $k$. For example, 
$G$ could be the multiplicative group $\dG_m$, or an abelian variety $A$. If 
$G$ is any commutative group scheme over $k$, we will use $\torsion n G$ to denote the 
fiber product $G\times_G 0$, where the map $G\to G$ is ``multiply by $n$.'' 
This conflicts with our earlier convention that $\torsion n A = \torsion n A(\bar k)$, but 
no confusion should arise from this. The group $G(\bar k)$ is 
a $G_k$-module, so we can consider the cohomology groups 
$\h^\bullet\left(G_k, G(\bar k)\right)$. We know that these are isomorphic to 
the \'etale cohomology groups 
$\h^\bullet\left(\et k, \widetilde{G(\bar k)}\right)$, and one can check quite 
easily that $\widetilde{G(\bar k)}$ is just $G$, regarded as a sheaf on $\et k$ 
via its functor of points. It follows that 
$\h^\bullet\left(G_k, G(\bar k)\right) = \h^\bullet(\et k, G)$, and we will 
identify the two without comment in the future. 

To simplify the notation, we will often write $k$ for $\et k$, i.e. 
$\h^\bullet(k,G) = \h^\bullet(\et k,G)$. In this context, Hilbert's 
\emph{Theorem 90} says that for $k$ a field, $\h^1(k,\dG_m) = 0$. If we write 
$\mu_n = \torsion{n}{\dG_m}$, then \emph{Kummer theory} starts with the short exact 
sequence $1\to \mu_n\to \dG_m\xrightarrow n \dG_m \to 1$ and uses Hilbert's 
Theorem 90 together with the long exact sequence in sheaf cohomology to derive 
$\h^1(k,\mu_n) = k^\times / n$. 

There is an alternate description of $\h^1(k,A)$. If $G$ is an arbitrary 
commutative algebraic group over $k$, a \emph{principal homogeneous space} 
(also called a \emph{torsor}) for $G$ over $k$ is a variety $X/k$ 
together with a morphism $G\times X\to X$ which, on $\bar k$-valued points, is 
a simply transitive group action. That is, if we write $g+x$ for the 
image of $(g,x)$ in $X$, we require that 
\begin{itemize}
  \item $g+(h+x) = (g+h)+x$ 
  \item $0+x = x$
  \item for all $x,y\in X(\bar k)$, there is a unique $g\in G(\bar k)$ such 
    that $g+x = y$
\end{itemize}
More generally, if $S$ is a scheme and $\sG$ is an abelian sheaf on $\et S$, a 
\emph{torsor} for $\sG$ is a sheaf of sets $\sT$ with a group action 
$\sG\times \sT\to\sT$ that, \'etale-locally on $S$, is isomorphic to $\sG$ as 
a sheaf with left $\sG$-action. Two torsors $\sT,\sT'$ are isomorphic if there 
is a sheaf isomorphism $\sT\to \sT'$ that commutes with the action of $\sG$. 
One can show (see e.g. \cite[IV 1.1]{de77}) that $\h^1(\et S,\sG)$ is 
naturally isomorphic as a pointed set to the set of isomorphism classes of 
$\sG$-torsors. 

Thus if $G$ is a commutative algebraic group over $k$, there is a natural 
bijection between the set of isomorphism classes of $G$-torsors and 
$\h^1(k,G)$. There is a non-abelian version of this. If $\sG$ is an arbitrary 
sheaf of groups over $\et S$, then one can still define the notion of a 
$\sG$-torsor. An identical theorem holds, except that one must define the 
``cohomology set'' $\h^1(\et S,\sG)$. For details, see \cite{sk01}. There is a 
non-abelian version of Hilbert's Theorem 90: it says that 
$\h^1(k,\text{SL}_n) = \h^1(k,\text{GL}_n) = 0$ for all $n$ \cite[X.1]{se79}. 










% class on 09-19-2013
\subsection{Selmer groups and weak Mordell-Weil}

Recall that if $G$ is a profinite group (e.g $G_k$ for some field $k$) and $M$ 
is a discrete abelian group with continuous $G$-action, we directly defined the 
first cohomology group $\h^1(G,M) = Z^1(G,M) / B^1(G,M)$, where 
\[
  Z^1(G,M) = \{\varphi:G\to M\text{ continuous}: \varphi(\sigma\tau) = \sigma \varphi(\tau)+\varphi(\sigma)\}
\]
and $B^1(G,M)$ consisted of all $\varphi:G\to M$ of the form 
$\sigma\mapsto \sigma x-x$. Note that if $G$ acts trivially on $M$, then 
$\h^1(G,M) = \hom_\text{cts}(G,M)$. If $f:M\to M'$ is $G$-equivariant, then 
there is an obvious map $f_*:\h^1(G,M)\to \h^1(G,M')$ given by 
$f_*[\varphi] = [f\circ \varphi]$. Moreover, if $f:G'\to G$ is a continuous 
group homomorphism, then we have a map $f^*:\h^1(G,M) \to \h^1(G',M)$ given 
by $f^*[\varphi] = \varphi\circ f$, where we regard $G'$ as acting on $M$ via 
$f$. 

Let $k$ be a number field, $A$ be an abelian variety over $k$, and 
$n\geqslant 2$. The short exact sequence $0\to \torsion n A\to A\xrightarrow n A \to 0$ 
of group schemes induces an exact sequence 
\[\xymatrix{
  0 \ar[r]  
    & A(k)/n \ar[r] 
    & \h^1(k,\torsion n A) \ar[r] 
    & \torsion{n}{\h^1(k,A)} \ar[r] 
    & 0 \text{.}
}\]
We are trying to prove that $A(k)/n$ is finite. Since $\h^1(k,\torsion n A)$ 
can be infinite, this does not follow immediately from the above exact 
sequence. 

\begin{example}
Let $K=\dQ$, and let $E$ be the elliptic curve defined by 
$y^2=(x-a)(x-b)(x-c)$ for distinct $a,b,c\in \dQ$. Then 
$\torsion 2 E(\bar\dQ) = \{0,(a,0),(b,0),(c,0)\}$. Thus 
\begin{align*}
  \h^1(\dQ,\torsion 2 E) &= \hom(G_\dQ,\torsion 2 E) \\
    &= \hom(G_\dQ,\dZ/2\times \dZ/2) \\
    &= \hom(G_\dQ,\dZ/2) \times \hom(G_\dQ,\dZ/2)
\end{align*}
This is easily seen to be infinite (either using global class field theory, or 
by noting that $\dQ$ has lots of Galois extensions of degree $2$). In fact, 
using Kummer theory, one can show that 
$\h^1(\dQ,\torsion 2 E) = (\dQ^\times/2)^{\oplus 2}$, which is $\dF_2$-vector space of 
dimension $\aleph_0$. 
\end{example}

Our goal is to find a finite subgroup of $\h^1(k,\torsion n A)$ containing $A(k)/n$. 
We haven't really used the fact that $k$ is a number field yet -- everything so 
far works for any perfect field. That changes when we start looking at 
completions of $k$. 

As before, let $k$ be a number field, and let $k_v$ denote the completion of 
$k$ at a place $v$. One can show that $k_v$ will either be $\dR$, $\dC$, or a 
finite extension of some $\dQ_p$. Choose $\overline{k_v}\supset k$; this gives 
a homomorphism $G_{k_v} \to G_k$, where $\sigma\mapsto \sigma|_{\bar k}$. It 
turns out that the map is injective (this is an easy corollary of Krasner's 
Lemma). As an example, 
if $k=\dQ$, $k_v=\dR$, then $G_{\dR}=\{1,c\}$ where $c:\dC\to \dC$ is complex 
conjugation. The image of $c$ in $G_\dQ$ is some element of order two. 

The functoriality of $\h^\bullet(-,-)$ applied to the injection $G_{k_v} \to G_k$ 
gives us a map $\h^1(k,\torsion n A) \to \h^1(k_v,\torsion n A)$, where we 
regard $\torsion n A$ as a 
group scheme over $k_v$ by base extension. We now have a commutative diagram: 
\[\xymatrix{
  0 \ar[r] 
    & A(k)/n \ar[r] \ar[d] 
    & \h^1(k,\torsion n A) \ar[r] \ar[d]^-\beta 
    & \torsion{n}{\h^1(k,A)} \ar[r] \ar[d] 
    & 0 \\
  0 \ar[r] 
    & \displaystyle\prod_v A(k_v)/n \ar[r]^-\alpha 
    & \displaystyle\prod_v \h^1(k_v,\torsion n A) \ar[r] 
    & \displaystyle\prod_v \torsion{n}{\h^1(k_v,A)} \ar[r] 
    & 0
}\]
The \emph{$n$-Selmer group} of $A$ over $k$ is 
\begin{align*}
  \selmer_n(A) = \beta^{-1}(\image \alpha) 
    = \ker\left(\h^1(k,\torsion n A) \to \prod_v \torsion{n}{\h^1(k_v,A)}\right) \text{.}
\end{align*}
Similarly, we define the \emph{Tate-Shafarevich group} of $A$ by the exact 
sequence 
\[\xymatrix{
  0 \ar[r] 
    & \sha(A) \ar[r] 
    & \h^1(k,A) \ar[r]
    & \displaystyle\prod_v \h^1(k_v,A) \text{.}
}\]
Putting these two definitions together, we obtain a short exact 
sequence 
\[\xymatrix{
  0 \ar[r] 
    & A(k)/n \ar[r] 
    & \selmer_n(A) \ar[r] 
    & \torsion{n}{\sha(A)} \ar[r] 
    & 0 \text{.}
}\]
We will soon prove that $\selmer_n(A)$ is finite, hence $A(k)/n$ and 
$\torsion{n}{\sha(A)}$ are finite. 

\begin{conjecture}[Tate-Shafarevich]
If $A$ is an abelian variety over a number field $k$, the group 
$\sha(A)$ is finite
\end{conjecture}

This conjecture is currently wide open. A positive answer would show that for $n$ 
sufficiently 
large, $A(k)/n \simeq \selmer_n(A)$. In particular, this would imply that for 
$p\gg 0$, $\rank(A) = \dim_{\dF_p} \selmer_p(A)$, where as before 
$\rank(A) = \rank_\dZ A(k)$ is the algebraic rank of 
$A$. 
The Selmer groups $\selmer_n(A)$ are effectively computable. If $\sha(A)$ were 
always finite, then $A(k)$ would be computable. 





\subsection{Crash course in algebraic number theory}

If $k$ is a number field, we already used (without defining it) the notion of a 
\emph{place} of $k$. In this section, we will see that $k$ comes with a lot of 
extra structure which will be used later on. Let's start with places. 

\begin{definition}
A \emph{local field} is a topological field that is locally compact as a 
topological space. 
\end{definition}

Let $k$ be a local field. Then the additive group of $k$ is a locally compact 
group, so it has a nontrivial \emph{Haar measure}, i.e. a 
translation-invariant Borel measure $\mu$. For $\alpha\in k$, the measure 
$\alpha^*\mu$ defined by $\alpha^* \mu(S) = \mu(\alpha S)$ is easily seen to 
be translation-invariant as well. It is known that $\mu$ is unique up to 
scalar, so there is a real number, denoted $|\alpha|$, such that 
$\alpha^*\mu = |\alpha|\mu$. It turns out that $|\cdot|$ induces the topology 
that $k$ already has, and that $k$ is complete with respect to $|\cdot|$. 
Unfortunately, $|\cdot|$ is not always an absolute value. That is, it may not 
satisfy 
\begin{enumerate}
  \item $|x|=0$ if and only if $x=0$
  \item $|x y| = |x|\cdot |y|$
  \item $|x+y|\leqslant |x|+|y|$
\end{enumerate}
However, the only exception is $k=\dC$, in which case $|\cdot |$ is the square 
of the usual absolute value. For any other local field, $|\cdot |$ is an honest 
absolute value, so we will speak of the ``canonical absolute value'' on a local 
field, even though it may not actually be an absolute value. 

Local fields can be completely classified. If the strict triangle inequality 
holds, i.e. $|x+y|\leqslant \sup\{|x|,|y|\}$ for all $x,y\in k$, we say that 
$k$ is \emph{non-archimedean}. If $k$ is not non-archimedean, we say it is 
\emph{archimedean}. If $k$ is archimedean, one can prove that $k$ has 
characteristic zero, so $\dQ\subset k$. The absolute value on $k$ induces one 
on $\dQ$, and it is a theorem that the only archimedean absolute value on 
$\dQ$ is the usual one. Thus $\dR\subset k$. It is a general theorem that $k$ 
can only be locally compact if $[k:\dR]<\infty$, from which it follows that 
either $k=\dR$ or $k=\dC$. 

If $k$ is non-archimedean of characteristic $p$, then one can prove that 
$k=\dF_q\lau t$ for some finite field $\dF_q$, where $q=p^r$. If $k$ is 
non-archimedean of characteristic zero, then once again $\dQ\subset k$. It is 
known (Otrowski's theorem) that the only non-archimedean absolute values on 
$\dQ$ are of the form $|\cdot|_p$ for primes $p$, where $|x|_p=p^{-v_p(x)}$. 
Here $v_p:\dQ^\times \to \dZ$ is the unique homomorphism with 
$v_p(p)=1$, $v_p(n)=0$ for $p\nmid n$. If we write $\dQ_p$ for the completion 
of $\dQ$ with respect to $|\cdot|_p$, then $k$ contains some $\dQ_p$. Once 
again a general theorem shows that $[k:\dQ_p]<\infty$. To summarize, local 
fields are one of the following:
\begin{itemize}
  \item $\dR$ or $\dC$
  \item $\dF_q\lau t$ for some prime power $q$
  \item finite extension of $\dQ_p$
\end{itemize}

If $k$ is a non-archimedean local field, we write 
$\fo_k=\{x\in k:|x|\leqslant 1\}$ and $\fp_k = \{x\in k:|x|<1\}$. It turns 
out that $\fo_k$ is a complete discrete valuation ring with maximal ideal 
$\fp_k$. We denote the residue field by $\kappa_k=\fo_k/\fp_k$. When $k$ is 
understood, we write $\fo,\fp,\kappa$ instead of $\fo_k,\fp_k,\kappa_k$. 
Choose a separable closure $k^s$ of $k$. Because the field $k$ is 
\emph{henselian} ($|\cdot|$ has a unique extension to $k^s$), the integral 
closure $\fo_{k^s}$ of $\fo_k$ in $k^s$ is a local ring (no longer noetherian) 
and elements of $G_k$ preserve $|\cdot|$. The field 
$\fo_{k^s}/\fp_{k^s}$ is separably closed, so we get a map 
$G_k\to G_\kappa$, given by $\sigma\mapsto \bar\sigma$, where 
$\bar\sigma(\bar x)=\overline{\sigma x}$. The kernel is called the 
\emph{inertia group} of $k$, and denoted $I_k$. 

\begin{definition}
A \emph{global field} is either a finite extension of either $\dQ$ or 
$\dF_p(t)$ for some prime $p$. 
\end{definition}
A \emph{place} of a global 
field $k$ is an equivalence class of embeddings $k\hookrightarrow K$, where 
$K$ is a local field such that the image of $k$ is dense. Two such 
embeddings are equivalent if they are (topologically) isomorphic over $k$. 
We will use the letter $v$ to denote places of $k$. Traditionally, a place of 
$k$ is defined to be an equivalence class of valuations -- the two definitions 
are equivalent. Any place $v$ induce a 
well-defined topology on $k$, and we will write $k_v$ for the completion of $k$ 
with respect to this topology. The completion $k_v$ is local. If $k_v$ is 
non-archimedean, we call $v$ \emph{finite}, otherwise it is \emph{infinite}. 
For the remainder, let $v$ be a finite place of $k$. 

The absolute value on $k_v$ induces one on $k$, which we will denote by 
$|\cdot|_v$. If $v$ is non-archimedean (i.e. $k_v$ is non-archimedean) then we 
will also use $v$ to denote the valuation on $k$ induced by the canonical 
valuation on $k_v$. We will write $\fo_v$, $\fp_v$, $\kappa_v$ instead of 
$\fo_{k_v}$\ldots. We can choose $k_v^s\supset k^s$, in which case restriction 
induces a continuous homomorphism $G_{k_v} \to G_k$. This is injective by 
Krasner's Lemma. The image is often denoted $D_v$, and the image of 
$I_{k_v}$ inside $D_v$ will be written $I_v$. Note that $D_v$, as a subgroup of 
$G_k$, is only well defined up to conjugacy. 

A useful fact about global fields is the \emph{product formula}. Given the 
canonical absolute value $|\cdot|_v$ 
associated with a place, we have 
\[
  \prod_v |x|_v = 1
\]
This property actually characterizes global fields -- see \cite{ar45}. For 
more details on local and global fields, see \cite{we95}. 



% class on 09-24-2013

Let $k$ be a number field, i.e. a finite extension of $\dQ$. Let $\fo_k$ be the 
\emph{ring of integers} of $k$, that is, $\fo_k$ is the integral closure of 
$\dZ$ in $k$. The ring $\fo_k$ is a dedekind domain. Thus if $\fa\subset \fo_k$ 
is a nonzero ideal, we have a factorization $\fa=\fp_1^{e_1}\dotsm \fp_r^{e_r}$ 
where the $\fp_i\subset\fo_k$ are nonz-zero prime (hence maximal) ideals, and each 
$e_i\geqslant 1$. This factorization is unique up to reordering if we require 
that the $\fp_i$ be distinct. 

For a prime $\fp\subset \fo_k$, we write (as before) $\kappa_\fp = \fo_k/\fp$ 
for the \emph{residue field} of $\fp$. The field $\kappa_\fp$ is finite because 
it is finitely generated (as a ring) over its prime field. There is a unique 
homomorphism $v_\fp:k^\times \twoheadrightarrow \dZ$ such that for 
$a\in \fo_k\smallsetminus 0$, we have $(a)=\fp^{v_\fp(a)}\cdot \fb$ with 
$(\fb,\fp)=1$. This gives us an absolute value 
\[
  |a|_\fp = \begin{cases}
              0 & \text{if $a=0$} \\
              (\# \kappa_\fp)^{-v_\fp(a)} & \text{otherwise}
            \end{cases}
\]
Completing $k$ with respect to this absolute value, we get a local field 
$k_\fp$. Our valuation (and absolute value) extend by continuity to a 
valuation $v_\fp:k_\fp^\times\to\dZ$ and absolute value 
$|\cdot|_\fp:k_\fp\to\dR_{\geqslant 0}$. This agrees with our previous 
definition of the canonical absolute value on a local field. 

Let $L/k$ be a finite extension, $\fp\subset \fo_k$ a prime ideal. The ideal 
$\fp\fo_L$ factors uniquely as 
$\fq_1^{e(\fq_1/\fp)}\dotsm\fq_r^{e(\fq_r/\fp)}$, where the $\fq_i\subset\fo_L$ 
are prime. For example, if $L=\dQ(i)$, then $2=-i(1+i)^2$ and $5=(1+2i)(1-2i)$. 
One can show that 
\begin{align*}
  2\fo_L &= ((1+i)\fo_L)^2 \\
  3\fo_L &= 3\fo_L \\
  5\fo_L &= ((1+2i)\fo_L)\cdot ((1-2i)\fo_L)
\end{align*}
are prime factorizations in $\fo_L$. One says that $\fp$ is \emph{unramified} 
in $L$ if $e(\fq_i/\fp)=1$ for all $i$. It is an easy theorem that only 
finitely many primes can ramify. 

Now assume $L/k$ is Galois. The action of the galois group $G=\gal(L/k)$ 
preserves $\fo_L$, fixing $\fo_k$. This action does \emph{not} preserve ideals in 
$\fo_L$. In fact, if $\fq_1,\dots,\fq_r$ are the primes lying above 
$\fp\subset \fo_k$, then $G$ acts transitively on $\{\fp_1,\dots,\fp_r\}$, 
from which we see that each $e(\fq_i/\fp)$ is the same integer, denoted 
$e_\fp$. Fix $\fq=\fq_1$. The \emph{decomposition group} of $\fq/\fp$ is 
\[
  D(\fq/\fp) = \{\sigma\in \gal(L/k) : \sigma(\fq) = \fq\}
\]
There is a canonical homomorphism 
$D(\fq/\fp) \to \gal(\kappa_\fq/\kappa_\fp)$ given by ``reduce $\sigma$ modulo 
$\fp$.'' This gives us an exact sequence 
\[\xymatrix{
  1 \ar[r] 
    & I(\fq/\fp) \ar[r] 
    & D(\fq/\fp) \ar[r] 
    & \gal(\kappa_\fq/\kappa_\fp) \ar[r] 
    & 1
}\]
(surjectivity on the right is non-trivial). So the \emph{inertia 
group} of $\fq/\fp$, denoted $I(\fq/\fp)$, is the subgroup of $D(\fq/\fp)$ 
consisting of automorphisms whose action is trivial modulo $\fq$. Choosing 
$\fq=\fq_i$ for some $i\ne 1$ gives a $D(\fq_i/\fp)$ that is conjugate to 
$D(\fq/\fp)$. We will often write $D_\fp$ and $I_\fp$, keeping in mind that 
they are only well-defined up to conjugacy. We will use the fact that 
$\# I_\fp = e_\fp$. So $\fp$ is unramified in $L$ if and only if 
$I_\fp = 1$. 

Choosing $\fq$ lying over $\fp$, we can complete $L$ and $k$ to get an 
extension $L_\fq/k_\fp$ of local fields. We have seen that restriction 
gives a map $\gal(L_\fq/k_\fp) \to \gal(L/k)$. It turns out that this map is an 
isomorphism onto the image $D(\fq/\fp)\subset \gal(L/k)$. We'll write 
$I(L_\fq/k_\fp)$ for the inverse image of $I(\fq/\fp)$ in 
$\gal(L_\fq/k_\fp)$. It can be defined directly in exactly the same manner as 
$I(\fq/\fp)$. 

Passing to the algebraic closure of $k_\fp$, we can consider the absolute 
Galois group 
\[
  G_{k_\fp} = \varprojlim_{L_\fq\supset k_\fp} \gal(L_\fq/k_\fp)
\]
where $L_\fq$ ranges over all finite Galois extensions of $k_\fp$. The group 
$G_{k_\fp}$ has a distinguished subgroup $I_\fp$, which is the inverse limit 
\[
  I_\fp = \varprojlim_{L_\fq/k_\fp} I(L_\fq/k_\fp) \text{.}
\]
Our exact sequence 
\[\xymatrix{
  1 \ar[r] 
    & I_\fp \ar[r] 
    & G_{k_\fp} \ar[r] 
    & \gal(\overline{\kappa_\fp}/\kappa_\fp) \simeq \hat\dZ \ar[r] 
    & 1
}\]
extends to a filtration of $G_{k_\fp}$ by normal closed subgroups whose 
successive quotients are abelian. Once again, recall that there is a canonical 
embedding $G_{k_\fp}\hookrightarrow G_k$ via an embedding 
$\bar k\hookrightarrow \overline{k_\fp}$. 










\subsection{Reduction of abelian varieties}

Let $k_\fp$ be a local field. Let $\fo_\fp$ be the 
ring of integers of $k_\fp$, i.e. $\fo_\fp=\{x\in k_\fp:|x|_\fp\leqslant 1\}$. 
The ideal $\fp\fo_\fp$ is the unique maximal ideal of $\fo_\fp$, and it turns 
out that $\fp\fo_\fp = \{x\in k_\fp:|x|_\fp<1\}$. In an abuse of notation, we 
write $\fp$ for $\fp\fo_\fp$. 

\begin{definition}
We say that a variety $X$ over $k_\fp$ has \emph{good reduction} if there is a 
smooth proper model $\cX$ of $X$ over $\fo_\fp$ .
\end{definition}
That is, there is a smooth proper scheme $\cX$ over $\fo_\fp$ such that the 
following diagram is cartesian:
\[\xymatrix{
  X \ar[r] \ar[d]  
    & \cX \ar[d] \\
  \spec(k_\fp) \ar[r] 
    & \spec(\fo_k)
}\]
We write $X_\fp$ for $\cX\times_{\spec(\fo_k)}\spec(\kappa_\fp)$, and call 
$X_\fp$ the \emph{reduction of $X$ modulo $\fp$}. 

One can show that if $A$ is an abelian variety over $k_\fp$ with good 
reduction, then $A_\fp$ is independent of the choice of $\cA$, that $\cA$ gets 
the structure of an abelian scheme over $\fo_k$, and that $A_\fp/\kappa_\fp$ is 
an abelian variety. 

There is in fact a canonical model for $A$ over $\fo_p$, and it exists in great 
generality. Let $S$ be a connected dedekind scheme with field of fractions $k$. 
If $A$ is an abelian variety over $k$, one calls a \emph{N\'eron model} of $A$ 
a smooth model $\cA$ for $A$ over $S$ for which any morphism $X_k\to A$, where 
$X$ is a smooth scheme over $S$, has a unique extension to a morphism 
$X\to \cA$. In other words, $\cA$ represents the functor 
$X\mapsto \hom_k(X_k,A)$ from smooth schemes over $S$ to $k$-varieties. It is 
clear that $\cA$ (if it exists) is unique up to unique isomorphism. 
Fortunately, it is a theorem (see \cite[1.4.1]{blr90}) that in our setting 
($S$ a connected dedekind scheme and $A$ an abelian variety over $k$) N\'eron 
models always exist. Since the functor $X\mapsto \hom_k(X_{\bar k},A)$ that 
$\cA$ represents is naturally a group-valued functor, $\cA$ naturally has the 
structure of a commutative group scheme \cite[1.2.6]{blr90}. 

So if $S=\spec(\fo_k)$ for a number field $k$ and $\fp\subset \fo$ is a prime, 
we could have said that an abelian variety $A/k$ has good reduction at $\fp$ 
if the N\'eron model $\cA$ for $\fo_\fp$ (the localization of $\fo$ at 
$\fp$) is proper. 

\begin{example}
Let $E:y^2=x^3+a x+b$, where $a,b\in \dZ$ and $\Delta=-16(4 a^3+27b^2)\ne 0$. 
If $p\nmid \Delta$, then $E/\dQ_p$ has good reduction, and $E_p/\dF_p$ is given 
by the reduction of our original equation modulo $p$. 
\end{example}

There is a reduction map $A(k_\fp)\to A_\fp(\kappa_\fp)$ which is a group 
homomorphism. This is the composite 
\[
  A(k_\fp)=\cA(\fo_\fp) \to \cA(\kappa_\fp) = A_\fp(\kappa_\fp)
\]
To see that $A(k_\fp)=\cA(\fo_\fp)$, think of $A$ as being a subset of some 
projective space $\dP^N$. For a point
$x=(x_0:\cdots:x_N)\in \dP^N(k_\fp)$, we can scale $x$ be the 
denominators of the $x_i$ to get a model 
$x=(x_0:\cdots:x_N)\in\dP^N(\fo_\fp)$. We can even get a model with 
some $x_i\not\equiv 0\mod\fp$, and the image of $x$ in 
$\dP^N(\kappa_\fp)$ is in $A_\fp(\kappa_\fp)$. By Hensel's lemma, the 
map $A(k_\fp) \to A_\fp(\kappa_\fp)$ is a surjection. The kernel is a pro-$p$ 
group, where $p$ is the characteristic of $\kappa_\fp$. In fact, the kernel is 
a $p$-adic Lie group. 

We can extend our reduction map to algebraic closures, getting a homomorphism 
$A(\overline{k_\fp}) \to A_\fp(\overline{\kappa_\fp})$, that has a pro-$p$ 
kernel. Choose an integer $n\geqslant 2$ with $p\nmid n$. The map 
$\torsion{n}{A(\overline{k_\fp})} \to \torsion{n}{A_\fp(\overline{\kappa_\fp})}$ 
is an isomorphism 
because both groups have the same cardinality ($n^{2\dim A}$) and the kernel is 
pro-$p$. In that isomorphism both groups have a Galois action -- $G_{k_\fp}$ on 
the left and $G_{\kappa_\fp}=\hat\dZ$ on the right. The map is compatible with 
the Galois action, so in particular, the inertia group $I_\fp$ acts trivially 
on $\torsion{n}{A(\overline{k_\fp})}$. 

Let $A/k_\fp$ be an abelian variety with good reduction. Recall we had a map 
$\delta:A(k_\fp)/n \hookrightarrow \h^1(k_\fp,\torsion n A)$ defined as 
follows. For $x\in A(k_\fp)$, choose $y\in A(\overline{k_\fp})$ such that 
$n\cdot y=x$. We define the $1$-cocycle $\varphi=\delta(x)$ by 
$\sigma\mapsto \sigma y - y$. For $\sigma\in I_\fp$, the elements 
$\sigma y$ and $y$ have the same image in $A_\fp(\overline{\kappa_\fp})$, hence  
$\sigma y - y\in \torsion{n}{A(\overline{k_\fp})}$ has trivial image modulo $\fp$. It 
follows that $\sigma y-y=0$ since it is an $n$-torsion point that is $0$ 
modulo $\fp$. This tells us that $\varphi(\sigma) = 0$ for all 
$\sigma\in I_\fp$, i.e. $\varphi(I_\fp) = 0$. 

\begin{lemma}
Let $A$ be an abelian variety over a number field $k$. With the above notation, 
$\varphi(I_\fp) = 0$ for all $\fp\subset \fo_k$ of good reduction for $A$. 
\end{lemma}

If we let $S$ be the finite set of primes for which $A$ has bad reduction, we 
will show that the group
\[
  \h_S^1(k,\torsion n A) = \{[\varphi]\in \h^1(k,\torsion n A) : \varphi(I_\fp) = 0\text{ for all } \fp\notin S\}
\]
is finite. 















\subsection{Restricted ramification}\label{sec:res-ram}

Let $k$ be a number field, and let $S$ be a finite set of primes of $k$. We 
defined, for a $G_k$-module $M$, the ``cohomology with restricted 
ramification'' 
\[
  \h_S^1(G_k,M) = \{[\varphi]\in \h^1(G,M) : \varphi(I_\fp) = 0\text{ for all }\fp\notin S\}
\]
It is better to interpret $\h_S^1(G_k,M)$ in terms of Galois cohomology. First 
a clarification: $\varphi(I_\fp)=0$ actually means that $\varphi(I_\fp)=0$ for 
\emph{any} choice of $I_\fp$. (Recall that the embeddings 
$G_{k_\fp}\hookrightarrow G_k$, and hence $I_\fp\hookrightarrow G_k$, are only 
well-defined up to conjugacy.) One easily checks that, for any 
$[\varphi]\in \h^1(G_k,M)$, the set 
$\ker(\varphi)=\{\sigma:\varphi(\sigma)=0\}$ is a subgroup of $G_k$. In other 
words, if $H\subset G_k$ denotes the (normal) subgroup generated by the 
$I_\fp$ for $\fp\notin S$, then 
\[
  \h_S^1(G_k,M) = \ker\left(\h^1(G_k,M) \to \h^1(H,M)\right) 
                = \h^1(G_k/H,M^H) \text{,}
\]
the second equality coming from the inflation-restriction sequence. 

The normal subgroup of $G_k$ generated by $\bigcup_{\fp\notin S} I_\fp$ is the 
Galois group of an extension $k_S/k$. One can prove that whenever $L/k$ is a 
finite extension unramified away from $S$, then $L\subset k_S$. We write 
$G_{k,S}=\gal(k_S/k) = G_k/G_{k_S}$, and call $G_{k,S}$ a Galois 
group \emph{with restricted ramification}. In the future, if $M$ is a 
$G_k$-module for which the action of $G_k$ on $M$ is unramified away from 
$S$, we will write $\h^1(G_{k,S},M)$ instead of $\h_S^1(G_k,M)$. 

\hard{
Just as one can interpret the 
absolute Galois group $G_k$ as the \'etale fundamental group 
$\pi_1(\spec k)$, the group $G_{k,S}$ can be interpreted as an \'etale 
fundamental group using the following theorem.

\begin{theorem}
Let $S$ be normal connected scheme with function field $k$. Then the group 
$\pi_1(S)$ is naturally isomorphic to $\gal(k_S/k)$, where $k_S$ is the 
composite of all finite extensions $L\subset k^s$ for which the normalization 
of $S$ in $L$ is \'etale over $S$. 
\end{theorem}
\begin{proof}
See \cite[5.4.9]{sz09}.
\end{proof}

Let $S$ be a finite set of primes of $k$, and write $\fo_{k,S}=S^{-1} \fo_k$. A 
field extension $K/k$ is unramified outside $S$ precisely when the integral 
closure of $\fo_{k,S}$ inside $K$ is \'etale over $\fo_{k,S}$. In other words, 
the above theorem shows that $G_{k,S} = \pi_1(\spec \fo_{o,S})$. In proving the 
weak Mordell-Weil theorem, we will need following important finiteness result. 
}

\begin{theorem}\label{thm:hermite-fund}
Let $S$ be a finite set of primes of a number field $k$. Then $G_{k,S}$ has 
only finitely many open subgroups of any given index. 
\end{theorem}
\begin{proof}
This is just a rewriting of \autoref{thm:hermite}. 
\end{proof}

\begin{theorem}[Hermite]\label{thm:hermite}
Let $k$ be a number field, $S$ a finite set of places of $\fo_k$. For a fixed 
integer $N\geqslant 1$, there is a finite extension $K\subset \bar k$ such that 
if $L\subset \bar k$ is an extension of $k$ unramified outside of $S$ with 
$[L:k]\leqslant N$, then $L\subset K$. Moreover, the minimal such $K$ is Galois 
over $k$ and unramified outside $S$. 
\end{theorem}
\begin{proof}
This follows easily from \cite[B.2.14]{bg06}. 
\end{proof}

Hermite's theorem has a huge generalization. Call a profinite group $G$ 
\emph{small} if it has only finitely many open subgroups of any given index. 
If $X$ is a connected scheme of finite type such that $X\to \spec(\dZ)$ has 
dense image, then $\pi_1(X)$ is small \cite[2.8]{hh09}. This is useful because 
of the following theorem. 

\begin{theorem}\label{thm:small-groups}
Let $G$ be a small group. Then $\h^1(G,M)$ is finite for all finite 
continuous $G$-modules $M$.
\end{theorem}
\begin{proof}
Since $M$ is finite, there is an open normal subgroup $N\subset G$ such that 
$N$ acts trivially on $M$. Recall the \emph{inflation-restriction exact 
sequence} is 
\[\xymatrix{
  0 \ar[r]
    & \h^1(G/N,M^N) \ar[r]^-{\text{inf}} 
    & \h^1(G,M) \ar[r]^-{\text{res}} 
    & \h^1(N,M) \text{,}
}\]
where the first arrow is 
$[\varphi]\mapsto [\varphi\circ (G\twoheadrightarrow G/N)]$ on cocycles, 
and the restriction map is $[\varphi]\mapsto [\varphi|_N]$. The group 
$\h^1(G/N,M^N)$ is finite because both $G/N$ and $M^N=M$ are finite, and 
$\h^1(N,M)=\hom_\text{cts}(N,M)$ is finite because $N$ is small. 
\end{proof}

If we are willing to consider $\h^1(G,M)$ for $M$ non-abelian, then the 
converse is true. 
















% lecture 09-26-2013
\subsection{Torsion and weak Mordell-Weil}

Let $A$ be an abelian variety over a number field $k$. Recall that we are 
trying to show that $A(k)$ is finitely generated. We have shown that it is 
sufficient to prove that the quotient $A(k)/n$ is finite for some $n\geqslant 2$. 
Once we know that $A(k)$ is finitely generated, we can write 
$A(k)=A(k)_\text{tors}\oplus \dZ\cdot x_1\oplus \cdots \oplus \dZ x_r$, where 
each $x_i$ is of infinite order. The algebraic rank $r=\rank A$ is 
very difficult to compute in general, but $A(k)_\text{tors}$ is computable. 

Choose a prime $\fp\subset \fo_k$ at which $A$ has good reduction (this is true 
for all but finitely many $\fp$). We have a reduction map 
$A(k_\fp)\to A_\fp(\kappa_\fp)$, which has pro-$p$ kernel. The group 
$A(k)_\text{tors}$ is contained in $A(k_\fp)$, so we can think about its image 
in $A_\fp(\kappa_\fp)$. Since $A(k)_\text{tors}$ is finite, the kernel of 
$A(k)_\text{tors}\to A_\fp(\kappa_\fp)$ is a $p$-group. Pick another prime 
$\fp'$ of good reduction for $A$, with residue characteristic $\ell\ne p$. The 
kernel of the map 
$A(k)_\text{tors}\to A_\fp(\kappa_\fp)\times A_{\fp'}(\kappa_{\fp'})$ is a 
$p$-group and a $\ell$-group, hence trivial, i.e. $A(k)_\text{tors}$ is a 
subgroup of $A_\fp(\kappa_\fp)\times A_{\fp'}(\kappa_{\fp'})$. (This gives us a 
way to compute $A(k)_\text{tors}$, because $A_\fp(\kappa_\fp)$ and 
$A_{\fp'}(\kappa_{\fp'})$ are computable. ($X(\kappa)$ for $X$ any projective 
variety over any finite field $\kappa$ is computable, for stupid reasons.)

\begin{example}
Let $E$ be the elliptic curve over $\dQ$ given by $y^2=x^3+3$. This has 
descriminant $\Delta=-2^4\cdot 3^5$, so $E$ has good reduction away from $2$ 
and $3$. We can compute 
\begin{align*}
  E_5(\dF_5) &= \{O,(1,\pm 2),(2,\pm 1),(3,0)\} \\
  \#E_7(\dF_7) &= 13
\end{align*}
The kernel of $E(\dQ)_\text{tors}\to E_5(\dF_5)$ is a $5$-group, and the kernel 
of $E(\dQ)_\text{tors}\to E_7(\dQ_7)$ is a $7$-group. From this, we know that 
$E(\dQ)_\text{tors}$ has no points of order $5$ or $7$. Thus  
$E(\dQ)_\text{tors}$ embeds into groups of oder $6$ and $13$, so it is the 
trivial group. Since $(1,2)\in E(\dQ)$ and $E(\dQ)_\text{tors}=0$, we know that 
$E(\dQ)$ is infinite. 
\end{example}

\begin{example}
Let $E/\dQ$ be the curve defined by $y^2 + y = x^3-x^2-10 x-20$. One can check 
that $E$ has good reduction away from $11$. Easy computations yield 
\begin{center}
  \begin{tabular}{c|c}
    $p$ & $  \# E_p(\dF_p)$ \\ \hline
    2   & 5  \\
    3   & 5  \\
    5   & 5  \\
    7   & 10 \\
    13  & 10
  \end{tabular}
\end{center}
This shows us that $E(\dQ)_\text{tors}$ is either $0$ or $\dZ/5$. In fact, it 
is the latter with $E(\dQ)_\text{tors}=\langle (5,5)\rangle$. 
\end{example}

Let's get back to the weak Mordell-Weil theorem. Let $A$ be an abelian variety 
over a number field $k$, and let $S$ be the (finite) set of primes $\fp$ for 
which $A$ has bad reduction at $\fp$. There are natural maps 
$\delta:A(k_\fp)/n \to\h^1(k_\fp,\torsion n A)$, so we get a commutative diagram with 
exact rows: 
\[\xymatrix{
  0 \ar[r] 
    & A(k)/n \ar[r]^-\delta \ar[d]
    & \h^1(k,\torsion n A) \ar[d] \\
  0 \ar[r]
    & A(k_\fp)/n \ar[r]^-\delta 
    & \h^1(k_\fp,\torsion n A)
}\]
Since $A$ has good reduction outside $S$, the group $G_{k,S}$ defined in 
\ref{sec:res-ram} acts on $A[n]$, so we can consider the cohomology group 
$\h^1(G_{k,S},\torsion n A)$. Moreover, the $n$-Selmer group 
$\selmer_n(A)$ sits inside $\h^1(G_{k,S},\torsion n A)$, so 
$A(k)/n\hookrightarrow \selmer_n(A)\subset \h^1(G_{k,S},\torsion n A)$. The group 
$\selmer_n(A)$ is a ``better approximation'' of $A(k)/n$ than $\h^1(G_{k,S},\torsion n A)$, 
but it is $\h^1(G_{k,S},\torsion n A)$ that we will actually show is finite. 

\begin{theorem}
Let $A$ be an abelian variety over a number field $k$, and let 
$S$ be the set of places of $k$ at which $A$ has bad reduction. Then the group 
$\h^1(G_{k,S},\torsion n A)$ is finite.
\end{theorem}
\begin{proof}
This follows immediately from the fact that $G_{k,S}$ is \emph{small}, coupled 
with \autoref{thm:small-groups}. 

Explicitly, we first choose $L=k(\torsion n A)\supset k$. Then $L/k$ 
is unramified outside $S$ and $G_L$ acts trivially on $\torsion n A$. Let $S'$ 
denote the set of places of $L$ lying above $S$. 
The inflation-restriction exact sequence 
\[\xymatrix{
  0 \ar[r] 
    & \h^1(\gal(L/k),\torsion n A) \ar[r] 
    & \h^1(G_{k,S},\torsion n A) \ar[r] 
    & \h^1(G_{L,S'},\torsion n A) \text{,} 
}\]
reduces the problem to showing that 
$\h^1(G_{L,S'},\torsion n A) = \hom(G_{L,S'},\torsion n A)$ is finite. So we 
may as well assume $\torsion n A\supset k$ in the first place, and try to prove 
that $\h^1(G_{k,S},\torsion n A) \simeq \hom(G_{k,S},(\dZ/n)^{\oplus 2 d})$ is 
finite, where $d=\dim A$. For any $f:G_{k,S} \to (\dZ/n)^{\oplus 2 d}$, the image of 
$f$ is the Galois group of an extension $L/k$ unramified outside $S$ with 
$[L:k]\leqslant n^{2 d}$. By Hermite's Theorem, there is a fixed finite Galois 
extension $K/k$ such that all $L\subset K$. Thus 
$\h^1(G_{k,S},\torsion n A)\hookrightarrow \hom(\gal(K/k),\dZ/n)^{\oplus 2 d}$, a 
finite group.
\end{proof} 

There is a geometric interpretation of the Tate-Shafarevich group $\sha(A)$, 
defined by the exact sequence 
\[\xymatrix{
  0 \ar[r] 
    & \sha(A) \ar[r] 
    & \h^1(k,A) \ar[r] 
    & \displaystyle\prod_v \h^1(k_v,A) .
 }\]
Recall that a \emph{torsor} of $A$ over $k$ is a nice variety $X/k$ with a 
simply transitive group action $A\times X\to X$ which is a morphism of 
$k$-varieties. In other words, we require $A(L)\times X(L)\to X(L)$ to be 
a simply transitive group action whenever $X(L)\ne\varnothing$. If 
$x\in X(L)$, we get an isomorphism $A_L\to X_L$ which is \emph{not} generally 
defined over $k$. Two torsors are equivalent if they have compatible group 
actions (i.e. if they are $A$-equivariantly isomorphic over $k$). A torsor is 
\emph{trivial} if it is equivalent to $A$ with the usual left action. It turns 
out that a torsor $X$ is trivial if and only if $X(k)\ne\varnothing$. 
There is a natural bijection 
\[
  \{\text{torsors over $A$}\}/\text{equivalence} \leftrightarrow \h^1(k,A) .
\]

\begin{example}
Let $C/k$ be a nice curve of genus $1$. (It could be that $C$ is not elliptic, 
for instance if $C(k)=\varnothing$.) Let $E=\jac C$; this is an elliptic curve. 
There is an isomorphism $C\to \picard_C^1$ given by $x\mapsto [x]$. The action 
$\picard_C^0\times \picard_C^1\to \picard_C^1$, induced by $(D_1,D_2)\mapsto D_1+D_2$, makes $C$ a torsor over $E$. 
\end{example}










% lecture on 10-01-2013
\subsection{Tate-Shafarevich groups}

Let $k$ be a number field, $A$ an abelian variety over $k$. Recall the 
Tate-Shafarevich group of $A$ is 
\[
  \sha(A) = \ker\left(\h^1(k,A) \to \prod_v \h^1(k_v,A)\right) \text{.}
\]
The \emph{Weil-Ch\^atelet group} of $A$, written $\operatorname{WC}(A)$, is 
the group of torsors over $A$ modulo equivalence. As we have seen, there is a 
bijection $\operatorname{WC}(A) = \h^1(k,A)$. If $k$ is a number field, then 
$\operatorname{WC}(A)$ is infinite, but if $k$ is finite then 
$\operatorname{WC}(A)=0$. We'll define the map 
$\operatorname{WC}(A)\to \h^1(k,A)$. Fix an $A$-torsor $X$. There exists some 
$L/k$ with $X(L)\ne\varnothing$. Choose $x\in X(L)$, and define a cocycle 
$\varphi:G_k\to A(\bar k)$ by $\sigma\mapsto \sigma x - x$, where 
$\sigma x- x$ is the unique point $a\in A(\bar k)$ such that $a+x=\sigma(x)$. 
The image of $X$ in $\h^1(k,A)$ is the cocycle $\varphi$. It isn't too hard to 
check that this map is well-defined. We can use this bijection to give 
$\operatorname{WC}(A)$ the structure of an abelian group -- alternatively one 
can put $[S]=[T]+[U]$ if there is a $A$-equivariant morphism 
$T\times U \to S$. 

This tells us that there is a natural bijection between $\sha(A)$ and the set 
of torsors $X$ of $A/k$, such that $X(k_v)\ne\varnothing$ for all places $v$ of 
$k$. This can be generalized. Let $X/k$ be a nice variety. We say that $X$ 
satisfies the \emph{Hasse principle} if $X(k_v)\ne\varnothing$ for all $v$ 
implies $X(k)\ne\varnothing$. So $\sha(A)$ classifies torsors over $A$ that 
do not satisfy the Hasse principle. Clearly, if $X(k)\ne\varnothing$, then 
$X$ satisfies the Hasse principle. Similarly, if $X(k_v)=\varnothing$ for some 
$v$, then $X$ satisfies the Hasse principle. 

\begin{example}[Selmer]
The plane curve $C\subset \dP_\dQ^2$ given by $3 x^3+4 y^3+5 z^3=0$ fails the 
Hasse principle. In other words, $C(\dQ_p)\ne \varnothing$ for all $p$, 
$C(\dR)\ne\varnothing$, but $C(\dQ)=\varnothing$. If we let $E=\jac C$, then 
we know that $\sha(E)\ne 0$. 
\end{example}

It turns out that the Hasse principle can be checked. That is, for a nice 
variety $X$ over $\dQ$ there is an algorithm to determine whether 
$X(\dQ_v)\ne\varnothing$ for all primes. This is because $X$ will have good 
reduction at all but a finite (computable!) set of primes. If $X$ has good 
reduction at $p$ with integral model $\mathcal X$, then 
$\mathcal X(\dF_p)\ne\varnothing$ by the Weil conjectures. Since $\mathcal X$ 
is smooth, Hensel's lemma lets us lift an element of $\mathcal X(\dF_p)$ to 
$\mathcal X(\dZ_p)\subset X(\dQ_p)$. For a prime $p$ at 
which $X$ has bad reduction, we can still pick an integral model $\mathcal X$. 
Either there will be an $n$ for which $\mathcal X(\dZ/p^n)=\varnothing$, in 
which case $X(\dQ_p)=\varnothing$, or elements of $\mathcal X(\dZ/p^n)$ for 
$n\gg 0$ will lift to $X(\dZ_p)$. Checking whether $X(\dR)=\varnothing$ is easy 
analysis. 

If $C$ is a nice curve of genus zero over a number field, then $C$ satisfies 
the Hasse principle \cite[3.4]{ca67}. This is essentially the 
Hasse-Minkowski theorem. 

\begin{example}[descent]
Let $E/\dQ$ be the elliptic curve defined by $y^2=(x-e_1)(x-e_2)(x-e_3)$ where 
the $e_i\in \dZ$ are distinct. It's easy to see that 
$\torsion 2 E=\{0,(e_1,0),(e_2,0),(e_3,0)\}\simeq (\dZ/2)^{\oplus 2}$. We use 
$\{(e_1,0),(e_2,0)\}$ as a $\dF_2$-basis for $\torsion 2 E$. We have 
\begin{align*}
  \h^1(\dQ,\torsion 2 E) &= \hom(G_\dQ,\torsion 2 E) \\
    &= \hom(G_\dQ,\dZ/2)^{\oplus 2} \\
    &= \left(\dQ^\times/2\right)^{\oplus 2}
\end{align*}
Here, as always, ``$\hom$'' denotes the group of continuous homomorphisms, and 
we write $\dQ^\times/2$ for $\dQ^\times/(\dQ^\times)^2$. The isomorphism 
$\hom(G_\dQ,\dZ/2) \simeq \dQ^\times/2$ comes from Kummer Theory. 
Given a homomorphism $\varphi:G_\dQ\to \dZ/2$, the group $\ker(\varphi)$ 
fixes a field $k=\dQ(\sqrt d)$ with $d\in \dQ^\times$. The equivalence class of 
$d$ in $\dQ^\times/2$ depends only on $\varphi$. The fact that 
$\varphi\mapsto d$ is a bijection is a restatement of the main theorem of 
Kummer Theory. 

The boundary morphism in group cohomology gives us a map 
$\delta:E(\dQ)/2 \hookrightarrow \h^1(\dQ,\torsion 2 E)$. The 
composite of $\delta$ with the isomorphism 
$\h^1(\dQ,\torsion 2 E)\isomorphism )=(\dQ^\times/2)^{\oplus 2}$ is quite explicit 
-- we have for $P=(x_0:x_1:1)$:
\[
  \delta(P) 
    = \begin{cases}
        (1,1)             & \text{if $x=0$} \\
        (x_0-e_1,x_0-e_2) & \text{if $x_0\notin \{e_2,e_3\}$} \\
        \left(\frac{e_1-e_3}{e_1-e_2},e_1-e_2\right) & \text{if $x_0=e_1$} \\
        \left(e_2-e_1,\frac{e_2-e_3}{e_2-e_1}\right) & \text{if $x_0=e_2$}
      \end{cases}
\]
Let $S$ be the set of primes that divide $2(e_1-e_3)(e_2-e_3)(e_2-e_1)$; this 
contains the set of primes at which $E$ has bad reduction. We know that the 
group $\h^1(G_{\dQ,S},\torsion 2 E)$ is finite. Even better, one can show that 
it is $\mathcal H^{\oplus 2}$, where 
\[
  \mathcal H = \left\{b\in \dQ^\times/2 : \text{$v_p(b)\equiv 0\pmod 2$ for $p\notin S$}\right\}
\]
The group $\mathcal H$ is generated by $S\cup\{-1\}$. This allows us to bound 
the rank of $E$. We know that $\dim_{\dF_2}(\mathcal H)\leqslant \# S+1$, so 
the fact that $E(\dQ)/2\hookrightarrow \mathcal H^{\oplus 2}$ implies 
$\rank(E) \leqslant 2(\# S+1)$. 

We know that $E(\dQ)/2\hookrightarrow \selmer_2(E)\subset \mathcal H^{\oplus 2}$, 
so we could get a better bound on $\rank(E)$ if we could compute 
the image of $\selmer_2(E)$ inside of $\mathcal H^{\oplus 2}$. Recall that there 
is a commutative diagram:
\[\xymatrix{
  & \h^1(G_{\dQ,S},\torsion 2 E) \ar[r] 
    & \h^1(\dQ,E) \\
  0 \ar[r] 
    & E(\dQ)/2 \ar[r] \ar@{^{(}->}[u] 
    & \selmer_2(E) \ar[r] \ar@{^{(}->}[u] 
    & \torsion{2}{\sha(E)} \ar[r]
    & 0
}\]
For a pair $b=(b_1,b_2)\in \mathcal H^{\oplus 2}$, we get an element of 
$\h^1(\dQ,E)$. This gives us a torsor $X_b$ of $E/\dQ$. Observe that 
$b\in \selmer_2(E)$ if and only if $X_b(\dQ_p)\ne \varnothing$ for all $p$ and 
$X_b(\dR)\ne\varnothing$. 

The curve $X_b$ can be computed. We have $X_b\subset \dP_\dQ^3$, a nice curve 
of genus one defined by 
\begin{align*}
  b_1 z_1^2 - b_2 z_2^2 &= (e_2-e_1)z_0^2 \\
  b_1 z_1^2 - b_1 b_2 z_3^2 &= (e_3 -e_1) z_0^2
\end{align*}
\end{example}

\begin{example}
Let $E/\dQ$ be the elliptic curve $y=x^3-x=x(x-1)(x+1)$. We'll pick 
$e_1=0$, $e_2=1$, $e_3=-1$. For our curve we have $S=\{2\}$, so $\mathcal H$ is 
the subgroup of $\dQ^\times/2$ generated by $\{-1,2\}$. The group 
$\mathcal H^{\oplus 2}$ has representatives 
$\{(\pm 1,\pm 1),(\pm 2,\pm 1),(\pm 1,\pm 2),(\pm 2,\pm 2)\}$. We know that for 
$b=(b_1,b_2)\in (\dQ^\times)^2$, the curve $X_b$ is 
\begin{align*}
  b_1 z_2^2 - b_2 z_2^2 &= z_0^2 \\
  b_1 z_1^2 - b_1 b_2 z_3^2&=-z_0^2 \\
\end{align*}
Thus $X_b(\dR)=\varnothing$ if $b_1<0$ and $b_2>0$, or if 
$b_1>0$ and $b_2<0$. That tells us that 
$\selmer_2(E)\subset \{(b_1,b_2)\in \mathcal H^{\oplus 2}:b_1 b_2>0\}$, a group of 
order eight with representatives $\{\pm (1,1),\pm (2,1),\pm (1,2),\pm (2,2)\}$. 

We have 
\[\xymatrix{
  0 \ar[r] 
    & E(\dQ)/2 \ar[r]^-\delta 
    & \selmer_2(E) \ar@{^{(}->}[r] 
    & \langle-(1,1),(2,1),(1,2)\rangle
}\]
Since $\torsion 2 E\subset E(\dQ)$, this tells us that $E(\dQ)/2$ has order at least 
four. We have $\delta(\torsion 2 E)=\{(1,1),(-1,-2),(-1,-1),(1,2)\}$. Consider 
$b=(2,1)$. Then $X_b$ is 
\begin{align*}
  2 z_1^2 - z_2^2 &= z_0^2 \\
  2 z_1^2 - 2 z_3^2 &= -z_0^2
\end{align*}
This curve has real points, so we only need to check $X_b(\dQ_2)$. As an 
exercise, show that $X_b(\dZ/4)=\varnothing$, which implies 
$X_b(\dQ_2)=\varnothing$, which shows that $b\notin \selmer_2(E)$, and thus 
$\# E(\dQ)/2=4$. Since $4=\#(\torsion 2 E)$, we know that $\rank(E)=0$. 
\end{example}










% notes on 10-03-2013
\subsection{Weil heights}\label{sec:heights}

Recall that to prove the Mordell-Weil theorem, we needed the weak Mordell-Weil 
theorem and a good height function. We gave an extremely terse introduction to 
heights earlier -- here we will do things more carefully. 

The idea is as follows. Let $X$ be a nice variety over a number field $k$. We 
want a function $H:X(k)\to \dR$ that measures the ``arithmetic complexity'' of 
a point. For example, $\frac 1 2$ and $\frac{100001}{200001}$ are very close in 
$\dR$, but we should think of the latter as being much more ``arithmetically 
complex.'' We would want $H$ to have properties arising from the geometry of 
$X$, and (this is very important), be such that the sets 
$\{x\in X(k):|H(x)|\leqslant c\}$ are finite for all $c$. 

Let's start with heights on projective space over $\dQ$. Consider a point 
$x=(x_0:\cdots:x_n)\in \dP^n(\dQ)$. After scaling by a rational number, we can 
assume that the $x_i\in \dZ$ and $\gcd(x_0,\dots,x_n)=1$. We set 
\[
  H_\dQ(x) = \sup\{|x_0|,\dots,|a_n|\}
\]
It is easy to see that this is well defined, but it doesn't work very well over 
a general number field. Returning to our example, we have 
$H_\dQ(\frac 1 2:1)=2$, while $H_\dQ(\frac{100001}{200001}:1)=200001$, which is 
much larger. 

Now let $k$ be a number field. Each finite place $v$ of $k$ is associated with 
a prime $\fp_v\subset \fo=\fo_k$. We set 
\[
  \|a\|_v = |a|_{\fp_v} 
        = \begin{cases}
            \#(\fo/\fp_v)^{-v_\fp(a)} & \text{if $a\ne 0$} \\
            0                         & \text{if $a=0$}
          \end{cases}
\]
We call $\|\cdot\|_v$ the canonical absolute value associated with $v$. If $v$ 
is a real place of $k$, i.e. $v$ corresponds with $i:k\hookrightarrow \dR$, we 
set $\|a\|_v = |i(a)|$. Similarly, if $v$ is a place corresponding to an 
embedding $i:k\hookrightarrow \dC$ with dense image, we set 
$\|a\|_v=|i(a)|^2$. Note that $\|\cdot\|_v$ is \emph{not} an absolute value 
because it doesn't satisfy the triangle inequality. Recall the following 

\begin{theorem}[product formula]
Let $k$ be a global field. For $a\in k^\times$, we have 
\[
  \prod_v \|a\|_v = 1
\]
\end{theorem}
\begin{proof}
We'll only look at $k=\dQ$. Take $a=\pm \prod_p p^{v_p(a)}\in \dQ^\times$. We 
have $\|a\|_p=p^{-v_p(a)}$ for each $p$, while $\|a\|_\infty = |a|$. Thus
\[
  \prod_v \|a\|_v = \|a\|_\infty\cdot \prod_p p^{-v_p(a)}  = \prod_p p^{v_p(a)}\cdot \prod_p p^{-v_p(a)} = 1
\]
For general $k/\dQ$, one uses the norm map $N:k\to \dQ$ to prove the product 
formula for $k$.
\end{proof}

For $k$ a general number field, we define $H_k:\dP^n(k)\to [1,\infty)$ by 
\[
  H_k(a_0:\cdots:a_n) = \prod_v \sup\{ \|a_0\|_v,\dots,\|a_n\|_v\}
\]
If $k=\dQ$ and the $a_i\in \dZ$ with $\gcd(a_0,\dots,a_n)=1$, then 
$\max\{|a_0|_p,\dots,|a_n|_p\}=1$. Thus 
$H_\dQ(a_0:\cdots:a_n)=\sup\{|a_0|,\dots,|a_n|\}$. The height $H_k$ is 
well-defined because of the product formula. For $c\in k^\times$, we have 
\begin{align*}
  \prod_v \sup\{\|ca_0\|_v,\dots,\|ca_n\|_v\} 
    &= \prod_v \|c\|_v \prod_v \sup\{\|a_0\|_v,\dots,\|a_n\|_v\} \\
    &= 1\cdot \prod_v \sup\{\|a_0\|_v,\dots,\|a_n\|_v\}
\end{align*}
which implies $H_k(c a_0:\cdots:c a_n)=H_k(a_0:\cdots:a_n)$. It turns out that 
for each $c\in \dR$, we have  
\[
  \#\{a\in \dP^n(k) : H_k(a)\leqslant c\} = O\left(c^{(n+1)[k:\dQ]}\right)
\]
In particular, the set on the left is finite. 

The \emph{absolute height} is a map $H:\dP^n(\bar\dQ)\to [1,\infty)$ is defined 
by $H(a) = H_k(a)^{[k:\dQ]^{-1}}$ for any number field $k$ with $a\in \dP^n(k)$. 
The sets $\{x\in \dP^n(\bar\dQ):H(a)\leqslant c\}$ can be infinite, but the 
sets 
\[
  \{x\in \dP^n(\bar\dQ):H(a)\leqslant c\text{ and } [\dQ(x):\dQ]\leqslant d\}
\]
are finite. The \emph{logarithmic absolute height} is the map 
$h=\log H:\dP^n(\bar\dQ)\to [0,\infty)$. 

Let $X$ be a nice variety over $k$. We would like to define a reasonable height 
function on $X$. One way to do this is to take an embedding $\phi:X\to \dP^n$ 
(probably for some very large $n$) and let $h_\phi:X(\bar k)\to [0,\infty)$ be 
the composite $X(k) \xrightarrow\phi \dP^n(\bar k) \xrightarrow{h} [0,\infty)$. 
We can rephrase this. Let $D\in \divisor(X)$ be a very ample divisor. A choice of a 
generating set for the global sections of $\sL(D)$ gives an embedding 
$\phi_D:X\hookrightarrow \dP^n$. We write $h_D$ for the 
composite 
$X(\bar k) \xrightarrow{\phi_D} \dP^n(\bar k)\xrightarrow{h} [0,\infty)$. Note 
that the notation $h_D$ is a little bit misleading, because $h_D$ actually 
depends on a choice of a generating set for $\sL(D)$. If we chose another 
generating set, getting an embedding $\phi_D'$ of $X$ into some projective 
space, then we have $h \phi_D - h \phi_D'=O(1)$, i.e. the two heights differ by 
a (globally) bounded function $X(\bar k)\to \dR$. So if we consider 
$h_D$ as an equivalence class in $\dR^{X(\bar k)}$, then $h_D$ is well-defined. 
If $E\sim D$ is another very ample divisor, then $h_D-h_E=O(1)$, so the 
equivalence class of $h_D$ only depends on the class of $D$ in the class group 
of $X$. Finally, if $E,D\in \divisor(X)$ are very ample, then 
$h_{D+E}=h_D+h_E+O(1)$. Sometimes we will write $h_{X,D}$ to emphasize the 
dependence of the height on $X$. 

\begin{theorem}[Weil's ``height machine'']\label{thm:weil-height}
Let $X$ be a nice variety over a number field $k$. Then there exists a unique 
homomorphism $h:\picard(X)\to \dR^{X(\bar k)}/O(1)$ such that 
\begin{enumerate}
  \item If $D$ is very ample, then $h_D=h\circ \phi_D+O(1)$. (normalization)
  \item For a morphism $\phi:X\to Y$ of nice $k$-varieties, we have 
    $h_{X,\phi^\ast D} = h_{Y,D}\circ\phi + O(1)$ for all $D\in \divisor(Y)$. 
    (functoriality)
  \item For all effective $D$, $h_D\geqslant O(1)$ on the complement of the 
    base locus of $D$. (positivity)
\end{enumerate}
\end{theorem}
\begin{proof}
The existence and uniqueness of $h$ follows easily from the above remarks. For 
a proof of 2 and 3, see \cite[2.3]{bg06}. 
\end{proof}





\subsection{N\'eron-Tate heights}\label{sec:neron-tate}

The case we are interested in is when $X=A$ is an abelian variety over $k$. For 
$D\in \divisor(A)$, one can show (see \cite[I.5.4]{mi-av}) that 
$[n]^\ast D\sim \frac{n(n+1)}{2} D + \frac{n(n-1)}{2}[-1]^\ast D$. In 
particular, if $D$ is symmetric (that is, $[-1]^\ast D=D$) then 
$[n]^\ast D\sim n^2 D$. (There are plenty of symmetric divisors: if $D$ is 
arbitrary, $D+[-1]^\ast D$ is symmetric). By \autoref{thm:weil-height}, 
if $D$ is symmetric, we have 
\begin{align*}
  h_{A,D}\circ[n]+O(1)
    &= h_{A,[n]^\ast D} + O(1) \\
    &= h_{A,n^2 D}+O(1) \\
    &= n^2 h_{A,D}+O(1)
\end{align*}
So for all $x\in A(\bar k)$, we have 
$h_{A,D}(n\cdot x) = n^2 h_{A,D}(x)+O(1)$. In other words, repeated addition 
gives a quadratic growth in the ``complexity'' of a point $x\in A(\bar k)$. 

Fix a symmetric divisor $D\in \divisor(A)$. Define the \emph{canonical (or 
N\'eron-Tate) height} associated to $D$, as a function 
$\widehat h_D:A(\bar k)\to \dR$, by 
\[
  \widehat h_D(a) = \lim_{n\to \infty} \frac{1}{4^n} h_D(2^n\cdot a) \text{.}
\]


% class on 10-08-2013
Let's show that the limit defining $\widehat h_D$ exists. Our previous 
discussion shows that $|h_D(2\cdot x)-4 h_D(x)|\leqslant C$ for some constant $C$ 
depending only on $A$ and $D$. We write this as $h_D(2\cdot x)=4 h_D(X)+O(1)$, 
keeping in mind that this $O(1)$ only depends on $A$ and $D$. Note that 
\[
  \frac{h_D(2^i\cdot x)}{4^i} = \frac{h_D(2\cdot (2^{i-1}\cdot x))}{4^i} = \frac{4 h_D(2^{i-1}\cdot x)+O(1)}{4^i} = \frac{h_D(2^{i-1}\cdot x)}{4^{i-1}} + \frac{O(1)}{4^i} \text{.}
\]
Thus, for any $n>m$, we have 
\begin{align*}
  \left| \frac{h_D(2^n \cdot x)}{4^n} - \frac{h_D(2^m\cdot x)}{4^m}\right|
    &\leqslant \sum_{i=m+1}^n \left|\frac{h_D(2^i\cdot x)}{4^i} - \frac{h_D(2^{i-1}\cdot x)}{4^{i-1}}\right| \\
    &= \sum_{i=m+1}^n \left|\frac{h_D(2^{i-1}\cdot x)}{2^{i-1}} +\frac{O(1)}{4^i} - \frac{h_D(2^{i-1}\cdot x)}{2^{i-1}}\right| \\
    &\leqslant \sum_{i=m+1}^n \frac{C}{4^i} \text{,}
\end{align*}
which shows that $\left\{4^{-n} h_D(2^n\cdot x)\right\}_{n=1}^\infty$ is a Cauchy 
sequence. So the limit defining $\widehat h_D(x)$ exists. If we set $m=0$ and 
let $n\to \infty$, we see that $h_D=\widehat h_D+O(1)$. 

If $D$ is antisymmetric, we define 
$\widehat h_D(x)=\lim_{n\to\infty} \frac{1}{2^n} h_D(2^n\cdot x)$
The same proof shows that $\widehat h_D$ is well-defined. We can extend 
$\widehat h$ to a function $\widehat h:\divisor(A)\to \dR^{A(\bar k)}$ by setting 
\[
  \widehat h_D=\frac 1 2 \left(\widehat h_{D+[-1]^\ast D} +  h_{D-[-1]^\ast D}\right) \text{.}
\]
It turns out that $\widehat h_D$ depends only on the class of $\sL(D)$ in 
$\picard(A)$, so we will think of $\widehat h$ as a function 
$\widehat h:\picard(A) \to \dR^{A(\bar k)}$. 

For $c\in \picard(A)$, the N\'eron-Tate height $\widehat h_c$ has many nice 
properties, for example 
\begin{align*}
  \widehat h_c(n\cdot x) &= n^2 \widehat h_c(x) \\
  \widehat h_c(x+y)+\widehat h_c(x-y) &= 2\left(\widehat h_c(x)+\widehat h_c(y)\right)
\end{align*}
The first is easy to show using properties of Weil heights, and the second is 
an easy consequence of the following lemma.

\begin{lemma}\label{lem:height-quadratic}
Let $A$ be an abelian variety over a number field $k$, and let $c\in \picard(A)$. 
Then the function 
$\langle \cdot,\cdot\rangle:A(\bar k)\times A(\bar k)\to \dR$ defined by 
\[
  \langle x,y\rangle = \frac 1 2 \left(\widehat h_c(x+y)-\widehat h_c(x)-\widehat h_c(y)\right)
\]
is bilinear. 
\end{lemma}
\begin{proof}
We are trying to prove that for $x,y,z\in A(\bar k)$, we have 
\[
  \widehat h_c(x+y+z)-\widehat h_c(x+y)-\widehat h_c(z) = \left(\widehat h_c(x+z)-\widehat h_c(x)-\widehat h_c(z)\right) + \left(\widehat h_c(y+z)-\widehat h_c(y)-\widehat h_c(z)\right) \text{.}
\]
This is equivalent to 
\begin{equation}\label{eq:height-quadratic}
  -\widehat h_c(x+y+z)+\widehat h_c(x+y)+\widehat h_c(x+z)+\widehat h_c(y+z)-\widehat h_c(x)-\widehat h_c(y)-\widehat h_c(z) = 0 \text{.}
\end{equation}
Now we use the fact that for a morphism $f:A\to B$ of abelian varieties, one has 
$\widehat h_{f^\ast c}=\widehat h_c\circ f$. For a finite set $S$ and 
$\sigma\subset S$, write $\pi_\sigma:A^S=\prod_S A\to A$ for the map 
$(x_s)_{s\in S}\mapsto \sum_{s\in\sigma} x_s$. With this, we can rewrite 
equation \eqref{eq:height-quadratic} as 
\[
  \sum_{\sigma\subset \{1,2,3\}} (-1)^{|\sigma|}\widehat h_c\circ \pi_\sigma = 0 \text{.}
\]
By the functoriality of $\widehat h$, it is enough to show that one has 
\[
  \sum_{\sigma\subset \{1,2,3\}} (-1)^{|\sigma|} \pi_\sigma^\ast = 0 \text{.}
\]
This is just the ``theorem of the cube'' (\autoref{thm:cube}). 
\end{proof}

The basic idea in this proof comes from the proofs of theorems 8.6.11 and 
9.2.8 in \cite{bg06}. 

\begin{theorem}\label{thm:cube}
Let $A$ be an abelian variety over a field $k$. For any finite set $S$ with 
$\# S\geqslant 3$, one has 
\[
  \sum_{\sigma\subset S} (-1)^{|\sigma|} \pi_\sigma^\ast = 0
\]
as a map $\picard(A)\to \picard(A^S)$. 
\end{theorem}
\begin{proof}
Notation is as in \autoref{lem:height-quadratic}. For $\# S=3$ this is 
just \cite[2.7]{gm13}. The general case follows easily by induction. 
\end{proof}



If $c$ is the class of a very ample line bundle, then 
$\widehat h_c\geqslant 0$ and for all $x\in A(\bar k)$, 
$\widehat h_c(x)=0$ if and only if $x$ is torsion. 
Indeed, if $x$ is torsion, then for some $n>1$, we have 
$n^2 \widehat h_c(x)=\widehat h_c(n x)=\widehat h_c(x)$, which implies 
$\widehat h_c(x)=0$. On the other hand, if $\widehat h_c(x)=0$, then 
$\widehat h_c(n\cdot x)=0$ for all $n$. The set 
$\{x,2\cdot x,3\cdot x,\dots\}$ is contained inside $A(L)$, where $L=k(x)$ is 
the finite extension of $k$ generated by (the coordinates of) $x$. Thus 
$\mathbb{N}\cdot x\subset \{a\in A(L):\widehat h_c(a)\leqslant 0\}$, which we 
know is a finite set. It follows that $x$ is torsion. 

\begin{example}
Let $E/\dQ$ be an elliptic curve. Then a divisor on $E$ is just a formal sum 
$\sum_{x\in E} n_x\cdot x$. Of course, $E$ has a canonical point -- the origin 
$O$, so for each $n\in \dZ$, we can consider the N\'eron-Tate height associated 
with $n O$. Since $\widehat h_{n O}=n \widehat h_O$, we don't lose any 
information by restricting ourselves to $n=1$, but since $O$ isn't very ample, 
we can only directly compute $h_{n O}$ starting with $n=2$. 

For $n=2$, the Riemann-Roch theorem tells us that $\ell(2 O)=2$, i.e. 
$\h^0(\sL(2 O)) = k\oplus k x$, where $x$ is a rational function on $E$. For 
$E$ with chosen model $y^2=x^3+a x+b$, $x$ the rational function corresponds to 
the map $E\to \dP^1$ given by $(x:y:1)\mapsto (x:1)$. Thus 
$h_{2 O}(x:y:1)=h_\dQ(x)$. 

For $n=3$, the divisor $3 O$ is actually very ample. By the Riemann-Roch 
theorem, $\ell(3 O)=3$, so $\h^0(\sL(3 O))=k\oplus k x\oplus k y$, where 
$(x:y:1):E\to \dP^2$ gives a Weierstrass embedding. That is, there exist 
$a,b$ such that $y^2=x^3+a x+b$, and $\phi_{3 O}:E\to \dP^2$ induces an 
isomorphism between $E$ and the zero set of $y^2=x^3+a x+b$ in $\dP^2$. 
One has $\widehat h_{3 O}(e)=h_\dQ(x(e):y(e):1)+O(1)$. 
\end{example}

Once again, let $c$ be the class of a symmetric, very ample line bundle on an 
abelian variety $A/k$. The function  
\[
  \langle x,y\rangle_c = \frac 1 2 \left(\widehat h_c(x+y) - \widehat h_c(x)-\widehat h_c(y)\right)
\]
is bilinear, and descends to $A(k)/A(k)_\text{tors}$. 
We can even tensor with $\dR$ to get a bilinear form on 
$A(k)_\dR = A(k)\otimes \dR$. The form $\langle \cdot,\cdot\rangle_c$ 
is positive-definite on $A(k)_\dR$. Thus $A(k)_\dR$ with the pairing 
$\langle\cdot\rangle_c$ is isomorphic to $\dR^n$ with the standard inner 
product, where $n=\rank A$. The group $A(k)/A(k)_\text{tors}$ is a 
lattice in $A(k)_\dR$, and we can define the \emph{regulator} of $A$ with 
respect to $c$ to be the square of the volume of the fundamental domain of 
$A(k)/A(k)_\text{tors}$ in $A(k)_\dR$. That is, 
\[
  \operatorname{Reg}_c(A) = \left|\det\left(\langle x_i,x_j\rangle_c\right)_{i,j}\right|
\]
where $\{x_i\}$ is a $\dZ$-basis for $A(k)/A(k)_\text{tors}$. 


If $A$ is an arbitrary abelian variety, there is no natural ``good choice'' for 
a canonical divisor on $A$. On the other hand, if $A=J=\jac C$ for some nice 
curve $C$ of genus $g$ with $C(k)\ne\varnothing$, then we can use the induced 
embedding $j:C\to J$ to define a canonical divisor on $J$. Let 
$\Theta=j(C)+\cdots + j(C)$ be the $(g-1)$-fold sum of $C$ in $J$, and consider 
$\Theta$ as a divisor on $J$. We call $\Theta$, and the induced class $\theta$ 
in $\picard(J)$, the \emph{theta-divisor of $J$}. Note that if $J=E$ is an elliptic 
curve with origin $O$, the theta-divisor is just $O$.  It turns out that 
$\theta$ is ample (see \cite[8.10.22]{bg06}), and thus 
$\langle\cdot,\cdot\rangle_\theta$ is positive-semidefinite. For any 
$c\in \picard(J)$ for which $\langle\cdot,\cdot\rangle_c$ is 
positive-semidefinite, set 
$|x|_c=\sqrt{\langle x,x\rangle_c} = \widehat h_c(x)^{1/2}$. The following 
theorem is very deep.

\begin{theorem}[Vojta's inequality]
Let $C$ be a nice curve of genus $g\geqslant 2$ over a number field $k$. Let 
$J=\jac C$, and let $\theta$ be the theta-divisor on $J$. Then there are 
effectively computable constants $\lambda,\lambda'$ depending only on $C$ and 
a chosen $x_0\in C(\bar k)$ such that for $x,y\in C\hookrightarrow J$, if 
$|x|_\theta\geqslant \lambda$ and 
$|y|_\theta\geqslant \lambda'|x|_\theta$, then 

\[
  \langle x,y\rangle_\theta \leqslant \frac 3 4 |x|_\theta\cdot |y|_\theta\text{.}
\]
\end{theorem}
\begin{proof}
This is hard -- see \cite[11.9.1]{bg06}. The constant $\frac 3 4$ is not 
important, as the theorem would hold for any constant in the interval 
$(g^{-1/2},1]$. 
\end{proof}










% notes on 10-10-2013

While the following theorem is due to Faltings, our proof is far easier than 
the original because it rests on deep properties of height functions. 

\begin{theorem}[Faltings]
Let $C$ be a nice curve of genus $g\geqslant 2$ over a number field $k$. Then 
$C(k)$ is finite.
\end{theorem}
\begin{proof}
We can assume $C(k)\ne\varnothing$, so the choice of $x_0\in C(k)$ gives us an 
embedding $j:C\hookrightarrow J=\jac C$ defined over $k$. Let 
$V=J(k)\otimes_\dZ\dR$. By Mordell-Weil, this is a finite-dimensional 
vector space, and its rank is just the (algebraic) rank of $J$. The induced 
pairing $\langle\cdot,\cdot\rangle_\theta$ on $V$ is now a nondegenerate 
positive-definite bilinear form, so $V$ is isomorphic as a Hilbert space to 
$\dR^n$ with the standard dot product. Note that by basic properties of 
heights, $S$ is a discrete subset of $V$. 

For nonzero $u,v\in V$, the \emph{angle} between $u$ and $v$ is the unique 
number $\varphi(u,v)$ in $[0,\pi]$ satisfying 
\[
  \cos\varphi(u,v) = \frac{\langle u,v\rangle_\theta}{|u|_\theta |v|_\theta} \text{.}
\]
For a nonzero $v$ and fixed $\alpha\in (0,\pi]$, define the cone 
$\Gamma_{v,\alpha} = \{u\in V\smallsetminus 0 : \varphi(u,v)<\alpha\}$. This is an open 
subset of $V$ that is stable under scaling by $\dR^+$. Let $S$ be the image of 
$C(k)$ in $J(k)/J(k)_\text{tors}\hookrightarrow V$. Since $J(k)_\text{tors}$ is 
finite, we only need to show that $S$ is finite. 

Let $\psi = \frac 1 2 \cos^{-1}(\frac 3 4)$, and for nonzero $v\in V$, let 
$U_v=\Gamma_{v,\psi}$. We claim that $U_v\cap S$ is finite. If it is not, 
then there exist $x,y\in S\cap U_v$  such that $|x|_\theta>\lambda$ and 
$|y|_\theta>\lambda'|x|_\theta$. By Vojta's theorem, 
$\cos\varphi(x,y)\leqslant \frac 3 4$, so 
$\varphi(x,y)\geqslant \cos^{-1}(\frac 3 4)$. This forces 
$\varphi(x,u)\geqslant \psi$, a contradiction. 

The sets $\{U_v:v\in V\smallsetminus 0\}$ form an open cover of $V$. Since the unit 
sphere $\{v\in V:|v|=1\}$ is compact, there is a finite list 
$v_1,\dots,v_n\in V$ such that $V=U_{v_1}\cup \cdots \cup U_{v_n}$. Since each 
$U_{v_i}\cap S$ is finite, the set $S$ is finite. 
\end{proof}

This proof gives an effectively computable upper bound for 
$\# C(k)$. Indeed, since the dimension of $V$ can be bounded above (for 
instance, by computing certain Selmer groups) it is possible to give an 
effectively computable upper bound for $\# C(k)$. On the other hand, the set 
$C(k)$ is not known to be effectively computable, because our proof does 
\emph{not} give a bound for the heights of points in $C(k)$.
